\subsection{Sprint Planning}
\label{subsec:sprint}

\indent Definimos la cantidad de horas del sprint basados en la discusión entre los distintos integrantes del grupo acerca de cuántas horas por semana podría cada uno dedicarle al trabajo práctico. Se decidió en 5 horas por semana por integrante, lo que se traduce en 20 horas por semana grupalmente. La duración del sprint es de 24 días, es decir, poco más de tres semanas y contabiliza aproximadamente 68 horas, de las cuales estimamos que el 10 \% se invierte en otros propósitos ajenos al desarrollo. Esto nos da un sprint de aproximadamente 62 horas.\\

\indent En la siguiente tabla ilustramos las stories que se definieron para el sprint, ordenadas de acuerdo a la relación entre business value y esfuerzo:\\

\begin{center}
  \begin{tabular}{| l | p{10cm} | l | l | }
    \hline
ID & Descripción & Business Value & Effort\\  \hline
US 17 & COMO administrador QUIERO que se repartan las fichas adecuadamente después de terminado el partido PARA calcular la nueva tabla de posiciones & 13 & 1\\  \hline
US 11 & COMO participante QUIERO apostar fichas PARA subir posiciones en la tabla & 8 & 2\\  \hline
US 1 & COMO participante QUIERO tener una cuenta PARA tener asociada mi información & 8 & 3\\  \hline
US 10 & COMO participante QUIERO poder crear y aceptar desafíos PARA medirme con otros participantes & 13 & 5\\  \hline
US 7 & COMO participante QUIERO conocer el libro de jugadas del técnico PARA saber cómo va a dirigir & 5 & 3\\  \hline
US 2 & COMO participante QUIERO armar un equipo PARA competir contra otros & 13 & 8\\  \hline
US 16 & COMO administrador QUIERO que quede un log y toda la información pertinente de cada simulación PARA que cualquier participante pueda consultarlo & 8 & 5\\  \hline
US 19 & COMO administrador QUIERO que el presupuesto de cada equipo no supere el cap del participante PARA equilibrar los valores de los equipos & 1 & 1\\  \hline
US 12 & COMO administrador QUIERO que los participantes puedan simular partidas PARA que jueguen entre sí & 21 & 21\\  \hline
  \end{tabular}
\end{center}



\begin{tcolorbox}
\subsubsection*{\underline{US17}}
\textbf{US17.TA1} \\
\textbf{Descripción}: Diseño de las ecuaciones de reparto del premio luego de las simulaciones.\\ 
\textbf{Duración estimada}: 0.5hs \\
\newline
\textbf{US17.TA2} \\
\textbf{Descripción}: Implementación del diseño con las adaptaciones correspondientes. \\
\textbf{Duración estimada}: 0.5hs \\
\newline
\textbf{US17.TA3} \\
\textbf{Descripción}: Ejecución de varias simulaciones, verificando la diferencia de fichas antes y después. \\
\textbf{Duración estimada}: 0.5hs 
\end{tcolorbox}
\vspace{10pt}



\begin{tcolorbox}
\subsubsection*{\underline{US11}}
\textbf{US11.TA1} \\
\textbf{Descripción}: Discusión y modelado sobre el mecanismo de apuestas. Decidir cuestiones implementativas.\\
\textbf{Duración estimada}: 1.0hs \\
\newline
\textbf{US11.TA2} \\
\textbf{Descripción}: Implementación del mecanismo de apuesta de fichas \\
\textbf{Duración estimada}: 1.0hs \\
\newline
\textbf{US11.TA3} \\
\textbf{Descripción}: Realizar casos de pruebas con valores de apuestas válidos e inválidos. Verificar que la apuesta quede asociada al participante y simulación correctos.\\
\textbf{Duración estimada}: 1.0hs
\end{tcolorbox}
\vspace{10pt}


\begin{tcolorbox}
\subsubsection*{\underline{US1}}
\textbf{US1.TA1} \\
\textbf{Descripción}: Discutir distintos tipos posibles de cuentas de usuario y definir el modelo que representará a los participantes en el sistema. Determinar la interfaz que permitirá ingresar datos.\\
\textbf{Duración estimada}: 1.0hs \\
\newline
\textbf{US1.TA2} \\
\textbf{Descripción}: Implementación de la abstracción que representa al usuario y de la pantalla de introducción de datos.\\
\textbf{Duración estimada}: 1.0hs \\
\newline
\textbf{US1.TA3} \\
\textbf{Descripción}: Implementación del mecanismo de log-in del sistema.\\
\textbf{Duración estimada}: 1.0hs \\
\newline
\textbf{US1.TA4} \\
\textbf{Descripción}: Creación y ejecución de casos de prueba tanto para registro como para login. Verificar que no se puedan crear usuarios inválidos. \\
\textbf{Duración estimada}: 1.0hs
\end{tcolorbox}
\vspace{10pt}


\begin{tcolorbox}
\subsubsection*{\underline{US10}}
\textbf{US10.TA1} \\
\textbf{Descripción}: Diseño del mecanismo de desafío. Discutir si los desafíos van a un pool de desafíos generales o si son dirigidos desde su creación.\\
\textbf{Duración estimada}: 2.0hs \\
\newline
\textbf{US10.TA2} \\
\textbf{Descripción}: Implementar desafíos y la pantalla de creación.\\
\textbf{Duración estimada}: 2.0hs \\
\newline
\textbf{US10.TA3} \\
\textbf{Descripción}: Implementar los mecanismos mediantes los cuales se aceptan o rechazan los desafíos.\\
\textbf{Duración estimada}: 1.0hs \\
\newline
\textbf{US10.TA4} \\
\textbf{Descripción}: Diseñar y ejecutar casos de prueba tanto para la creación de los desafíos como para su aceptación o rechazo.\\
\textbf{Duración estimada}: 1.0hs
\end{tcolorbox}
\vspace{10pt}



\begin{tcolorbox}
\subsubsection*{\underline{US7}}
\textbf{US7.TA1} \\
\textbf{Descripción}: Discusión sobre y modelado de la pantalla que permite a un participante investigar los libros de jugadas de cada técnico.\\
\textbf{Duración estimada}: 1.0hs \\
\newline
\textbf{US7.TA2} \\
\textbf{Descripción}: Implementar la pantalla con los técnicos teniendo en cuenta la existencia de diferentes filtros de consulta, además de ordenamiento de datos.\\
\textbf{Duración estimada}: 1.0hs \\
\newline
\textbf{US7.TA3} \\
\textbf{Descripción}: Implementar la pantalla con las jugadas de un técnico, teniendo en cuenta la existencia de diferentes filtros de consulta.\\
\textbf{Duración estimada}: 1.0hs \\
\newline
\textbf{US7.TA4} \\
\textbf{Descripción}: Diseño y ejecución de casos de prueba que permitan verificar que la información provista en las pantallas sea correcta y completa de acuerdo a la existencia o no de filtros.\\
\textbf{Duración estimada}: 1.0hs
\end{tcolorbox}
\vspace{10pt}


\begin{tcolorbox}
\subsubsection*{\underline{US2}}
\textbf{US2.TA1} \\
\textbf{Descripción}: Involucra la discusión y modelado de un equipo en el sistema, teniendo en cuenta las restricciones como el cap.\\
\textbf{Duración estimada}: 3.0hs \\
\newline
\textbf{US2.TA2} \\
\textbf{Descripción}: Implementar la pantalla de creación de equipos, así como los distintos componentes del sistema que conforman a la representación del equipo.\\
\textbf{Duración estimada}: 4.0hs \\
\newline
\textbf{US2.TA3} \\
\textbf{Descripción}: Implementar el mecanismo de guardado de equipos, y de las distintas validaciones que deben cumplirse.\\
\textbf{Duración estimada}: 1.0hs \\
\newline
\textbf{US2.TA4} \\
\textbf{Descripción}: Diseño y ejecución de casos de prueba que contemplen las distintas posibilidades a la hora de cargar equipos.\\
\textbf{Duración estimada}: 2.0hs
\end{tcolorbox}
\vspace{10pt}


\begin{tcolorbox}
\subsubsection*{\underline{US16}}
\textbf{US16.TA1} \\
\textbf{Descripción}:  Decisión de cómo mostrarle el log al usuario (¿archivo de output? ¿interfaz gráfica?), y cómo estructurar la información a mostrar. \\
\textbf{Duración estimada}: 1.5hs \\
\newline
\textbf{US16.TA2} \\
\textbf{Descripción}: Hacer que el output de cada ecuación de la simulación (resolución de cada una de las acciones; las jugadas elegidas) vaya al log. Traducirla a un formato de texto legible para un usuario común.\\
\textbf{Duración estimada}: 3.0hs \\
\newline
\textbf{US16.TA3} \\
\textbf{Descripción}: Realizar varias simulaciones, y ver que el resultado de las mismas se plasme correctamnete en el log.\\
\textbf{Duración estimada}: 1.5hs
\end{tcolorbox}
\vspace{10pt}


\begin{tcolorbox}
\subsubsection*{\underline{US19}}
\textbf{US19.TA1} \\
\textbf{Descripción}: Utilizar la inecuación (sumatoriaValores $<$ capParticipante), definir los mensajes de error y cómo mostrárselos al usuario.\\ 
\textbf{Duración estimada}: 0.5hs \\
\newline
\textbf{US19.TA2} \\
\textbf{Descripción}: Asignarle los valores a la inecuación cada vez que estén los 5 jugadores elegidos. \\
\textbf{Duración estimada}: 1.0hs \\
\newline
\textbf{US19.TA3} \\
\textbf{Descripción}: Probar todos los casos de la inecuación (casos: menor, mayor, igual) y realizar una selección de jugadores que de como resultado cada uno de ellos. Verificar que los mensajes de error sean correctos, y que no se deje formar un equipo que no cumpla la condición.
\textbf{Duración estimada}: 1.0hs
\end{tcolorbox}
\vspace{10pt}



\begin{tcolorbox}
\subsubsection*{\underline{US12}}
\textbf{US12.TA1} \\
\textbf{Descripción}:Involucra el análisis y el modelado de los distintos aspectos de la simulación, así como de las diferentes colaboraciones que se llevan a cabo entre los componentes del sistema para que la simulación se realice correctamente.\\
\textbf{Duración estimada}: 10 hs \\
\newline
\textbf{US12.TA2} \\
\textbf{Descripción}: Supone la implementación del diseño de la simulación.\\
\textbf{Duración estimada}: 10 hs \\
\newline
\textbf{US12.TA3} \\
\textbf{Descripción}: Definir y ejecutar casos de prueba que permitan verificar el correcto funcionamiento de cada una de las componentes de la simulación, tales como la resolución de jugadas, cantidad de turnos de desempate, equipos que participan y el uso correcto de sus estadísticas, y que el resultado de la simulación sea el deducido de los resultados de cada jugada.\\
\textbf{Duración estimada}: 5hs
\end{tcolorbox}
\vspace{10pt}

\subsection{Total de la iteración}

La suma de las horas de todas las tareas que entran en el Sprint es 62, que coincide con el tamaño que definimos para éste.

La velocity de un sprint es la cantidad de story points completados. En este caso es igual a 90. Este valor será más útil cuando hayan pasado más sprints y podamos compararlos.