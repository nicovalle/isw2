\subsection{Sprint Planning}

\indent En la siguiente tabla ilustramos las stories que se definieron para el sprint, ordenadas de acuerdo a la relación entre business value y esfuerzo:\\

\begin{center}
  \begin{tabular}{| l | p{10cm} | l | l | }
    \hline
ID & Descripción & Business Value & Effort\\  \hline
US 17 & COMO administrador QUIERO que se repartan las fichas adecuadamente después de terminado el partido PARA calcular la nueva tabla de posiciones & 13 & 1\\  \hline
US 11 & COMO participante QUIERO apostar fichas PARA subir posiciones en la tabla & 8 & 2\\  \hline
US 1 & COMO participante QUIERO tener una cuenta PARA usar el sistema y tener asociada mi información & 8 & 3\\  \hline
US 10 & COMO participante QUIERO poder crear y aceptar desafíos PARA medirme con otros participantes & 13 & 5\\  \hline
US 7 & COMO participante QUIERO conocer el libro de jugadas del técnico PARA saber como va a dirigir & 5 & 3\\  \hline
US 2 & COMO participante QUIERO armar un equipo (jugadores, jugador estrella y técnico) PARA competir contra otros & 13 & 8\\  \hline
US 16 & COMO administrador QUIERO que quede un log y toda la información pertinente de cada simulación PARA que cualquier participante pueda consultarlo & 8 & 5\\  \hline
US 19 & COMO administrador QUIERO que el presupuesto de cada equipo no supere el cap del participante PARA equilibrar los valores de los equipos. & 1 & 1\\  \hline
US 12 & COMO administrador QUIERO que los participantes puedan simular partidas PARA que jueguen entre sí & 21 & 21\\  \hline
  \end{tabular}
\end{center}

