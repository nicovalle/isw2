\section{Introducción}
\indent \indent Esta primera etapa del trabajo consistió en la planificación, utilizando la metodología ágil Scrum, del desarrollo de un simulador de partidos de básquet de fantasía.

\subsection*{User Stories}
Por las características del sistema pedido, donde hay una simulación -que representa una etapa del desarrollo bastante \emph{pesada}-, quedaron pocas user stories en total, pues gran parte del sistema se concentra específicamente en ese punto. Esto fue algo que se habló con el product owner y estaba dentro de lo esperable.

Justamente por ser algo que llevaba tanto tiempo, se estuvo en la duda de si convenía o no dividir la \textbf{User Story} correspondiente a la simulación. Una solución propuesta fue especificar en las user stories que una simulación se componía de turnos, éstos de jugadas, y éstas de acciones de los jugadores. Una división de ese modo resultó exagerada, y además iría en contra del principio de independencia para las user stories de \textbf{Scrum}; dado que la simulación debería entrar completa en un único sprint, por lo tanto, se decidió dejarla como una única \textbf{User Story}.

\subsubsection*{Valuación User Stories}
Tanto para la sección de \emph{Business Value} como de \emph{Effort} de cada User Story se decidió realizar poker planning entre los 4 integrantes del grupo. Cuando había discrepancias, se esgrimían los argumentos por los que cada uno había puesto el puntaje correspondiente, de manera de intentar convencer a los otros y así converger los criterios.

\subsection*{Roles}
Otro punto donde hubo dudas fue en cuanto a los roles. A primera vista, no parecería haber nadie más que participe del sistema más que el usuario final, a quien llamamos un \emph{participante}. 

Leyendo con un poco más de atención y por cómo se encontraban redactados algunos puntos específicos del enunciado, dejando algunas cosas abiertas con la posibilidad de que sufran modificaciones a futuro, nos pareció propicio considerar un rol de alguien que se encarga de ``mantener'' y administrar el sistema -que seguramente no sea el \emph{dueño}, aunque sí puede que esté dirigido por el mismo-. Ese sería el rol del \emph{administrador}.
\begin{itemize}
\item \textbf{Participante}: es quien se encarga de crear equipos, desafiar a otros participantes y participar de las simulaciones. El \textit{usuario final} del sistema.
\item \textbf{Administrador}: es aquel que actualiza los datos de los jugadores, define las jugadas de cada técnico y las configuraciones de la simulación, tales como la cantidad de turnos de cada una. Es un supervisor del sistema, quien lo regula, el que realiza las acciones para hacerlo más atractivo para los participantes, y más equilibrado.
\end{itemize}

\subsection{Sprint}
La duración del Sprint se decidió que sea de alrededor poco más de 3 semanas, es decir la totalidad del tiempo asignado, y la cantidad de horas hombre que irían en ella de 62 horas . Se llegó a este total dada una estimación de 5 horas semanales por integrante para el desarrollo, teniendo en cuenta horas que se desperdician no cumpliendo el desarrollo principal.\\


