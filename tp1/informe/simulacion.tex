\section{Organización de la Simulación}
La implementación de la demo de la simulación se organiza de la siguiente manera:
\begin{itemize}
 \item Un archivo main.py donde se cargan los datos de prueba y se inicia una simulación.
 \item Cuatro carpetas de aquellas clases que tienen alguna jerarquización (acciones, jugadas, resolvedores y posiciones), junto a sus clases herederas.
 \item El resto de las clases necesarias para correr una simulación.
\end{itemize}

Para correr una simulación, deberá ejecutarse el comando $python$ $main.py$ en una consola. Es posible ingresar previamente al archivo para modificar algunas estadísticas (dadas de alta de forma manual), el nombre de los equipos o de los jugadores, etc. y personalizar la salida por consola.

Además, se pueden modificar las ecuaciones utilizadas para calcular el éxito o fracaso de una acción, ingresando en la carpeta resolvedores y alterando los cálculos planteados en la función esExitoso() de cada uno de las clases correspondientes. 
