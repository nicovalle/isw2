\subsection{Product Backlog}
\indent En la siguiente tabla se encuentran todos los user stories definidos para el proyecto.\\

\begin{center}
  \begin{tabular}{| l | p{10cm} | l | l | }
    \hline
ID & Descripción & Business Value & Effort\\  \hline
US 1 & COMO participante QUIERO tener una cuenta PARA usar el sistema y tener asociada mi información & 8 & 3\\  \hline
US 2 & COMO participante QUIERO armar un equipo (jugadores, jugador estrella y técnico) PARA competir contra otros & 13 & 8\\  \hline
US 3 & COMO participante QUIERO tener una lista de mis equipos ya armados PARA ahorrar tiempo & 1 & 3\\  \hline
US 4 & COMO administrador QUIERO definir cuáles son los jugadores PARA que los participantes armen sus equipos & 8 & 5\\  \hline
US 5 & COMO administrador QUIERO poder actualizar las estadísticas y datos de los jugadores PARA ser fiel a la realidad & 5 & 5\\  \hline
US 6 & COMO administrador QUIERO poder actualizar los jugadores con datos reales utilizando algún servicio externo PARA que sea más fácil & 5 & 13\\  \hline
US 7 & COMO participante QUIERO conocer el libro de jugadas del técnico PARA saber como va a dirigir & 5 & 3\\  \hline
US 8 & COMO administrador QUIERO definir cuáles son los técnicos disponibles PARA que los participantes armen sus equipos & 5 & 3\\  \hline
US 9 & COMO administrador QUIERO poder definir las jugadas disponibles de los técnicos PARA enriquecer la simulación & 5 & 5\\  \hline
US 10 & COMO participante QUIERO poder crear y aceptar desafíos PARA medirme con otros participantes & 13 & 5\\  \hline
US 11 & COMO participante QUIERO apostar fichas PARA subir posiciones en la tabla & 8 & 2\\  \hline
US 12 & COMO administrador QUIERO que los participantes puedan simular partidas PARA que jueguen entre sí & 21 & 21\\  \hline
US 13 & COMO administrador QUIERO poder ajustar la duración (en turnos) de las simulaciones PARA que el sistema sea flexible & 3 & 3\\  \hline
US 14 & COMO administrador QUIERO poder modificar las fórmulas de resolución de acciones PARA ir ajustando el sistema a lo largo del tiempo & 8 & 8\\  \hline
US 15 & COMO administrador QUIERO que el primer turno de cada simulación sea al azar PARA hacerlo justo & 2 & 1\\  \hline
US 16 & COMO administrador QUIERO que quede un log y toda la información pertinente de cada simulación PARA que cualquier participante pueda consultarlo & 8 & 5\\  \hline
US 17 & COMO administrador QUIERO que se repartan las fichas adecuadamente después de terminado el partido PARA calcular la nueva tabla de posiciones & 13 & 1\\  \hline
US 18 & COMO participante QUIERO ver la tabla de posiciones PARA compararme con los otros jugadores & 5 & 5\\  \hline
US 19 & COMO administrador QUIERO que el presupuesto de cada equipo no supere el cap del participante PARA equilibrar los valores de los equipos. & 1 & 1\\  \hline
  \end{tabular}
\end{center}

Los criterios de aceptación de cada user story se encuentran a continuación.


\begin{tcolorbox}
\textbf{US 1}: COMO participante QUIERO tener una cuenta PARA usar el sistema y tener asociada mi información

\vline

\textbf{Criterios de aceptación:}
\begin{itemize}
 \item El participante puede ingresar sus datos (nombre y contraseña) en un formulario
 \item Si los datos son correctos, el participante accede al sitio siempre con su cuenta
 \item Si son incorrectos, vuelve al formulario y se le da un mensaje
 \item El mensaje incluye la posibilidad de que recupere la contraseña olvidada o que se registre
 \item Cuando se encuentra loggeado, el usuario verá su información asociada (equipos armados, posición en la tabla, cantidad de fichas, cap, etc.)
\end{itemize}
\end{tcolorbox}
\vspace{10pt}

\begin{tcolorbox}
\textbf{US 2}: COMO participante QUIERO armar un equipo (jugadores, jugador estrella y técnico) PARA competir contra otros

\vline

\textbf{Criterios de aceptación:}
\begin{itemize}
\item El participante accede a la lista de jugadores, y puede ordenar a los jugadores por nombre, precio, o estadística (FG, 3P, RPG, APG, BPG, SPG, TO, PPG, y altura)
\item No se puede terminar de armar un equipo sin haber elegido 5 jugadores, uno para cada posición 
\item No se puede elegir 2 veces al mismo jugador
\item Mientras se eligen jugadores, se puede ver la conformación temporal del equipo, junto al costo total del mismo
\item Si el precio del equipo es mayor al cap del participante, se muestra un mensaje de error
\item Una vez armado el equipo, se elige un jugador estrella entre esos 5
\item Una vez elegido el jugador estrella, se muestran la lista de técnicos y el detalle de su libro de jugadas
\item Se puede ordenar a los técnicos en base a sus gustos (frecuencia asociada a cada jugada)
\end{itemize}
\end{tcolorbox}
\vspace{10pt}

\begin{tcolorbox}
\textbf{US 3}: COMO participante QUIERO tener una lista de mis equipos ya armados PARA ahorrar tiempo

\vline

\textbf{Criterios de aceptación:}
\begin{itemize}
\item El participante no verá equipos de otras personas
\item No faltará ningún equipo ya usado por el jugador
\item Se podrá ver el registro de partidos ganados y perdidos de cada equipo pasado
\end{itemize}
\end{tcolorbox}
\vspace{10pt}


\begin{tcolorbox}
\textbf{US 4}: COMO administrador QUIERO definir cuáles son los jugadores PARA que los participantes armen sus equipos

\vline

\textbf{Criterios de aceptación:}
\begin{itemize}
\item No se podrá ingresar ningún jugador dos veces 
\item Todos los jugadores deberán tener todas las estadísticas completas 
\item Si se consultan los jugadores, aparecen todos los cambios correctamente 
\end{itemize}
\end{tcolorbox}
\vspace{10pt}


\begin{tcolorbox}
\textbf{US 5}: COMO administrador QUIERO poder actualizar las estadísticas y datos de los jugadores PARA ser fiel a la realidad

\vline

\textbf{Criterios de aceptación:}
\begin{itemize}
\item El jugador ya debe estar en el sistema 
\item Si se consultan los datos de los jugadores, reflejarán la actualización
\end{itemize}
\end{tcolorbox}
\vspace{10pt}


\begin{tcolorbox}
\textbf{US 6}: COMO administrador QUIERO poder actualizar los jugadores con datos reales utilizando algún servicio externo PARA que sea más fácil

\vline

\textbf{Criterios de aceptación:}
\begin{itemize}
\item %CA
\item %CA
\item %CA
\end{itemize}
\end{tcolorbox}
\vspace{10pt}


\begin{tcolorbox}
\textbf{US 7}: COMO participante QUIERO conocer el libro de jugadas del técnico PARA saber como va a dirigir

\vline

\textbf{Criterios de aceptación:}
\begin{itemize}
\item %CA
\item %CA
\item %CA
\end{itemize}
\end{tcolorbox}
\vspace{10pt}


\begin{tcolorbox}
\textbf{US 8}: COMO administrador QUIERO definir cuáles son los técnicos disponibles PARA que los participantes armen sus equipos

\vline

\textbf{Criterios de aceptación:}
\begin{itemize}
\item %CA
\item %CA
\item %CA
\end{itemize}
\end{tcolorbox}
\vspace{10pt}


\begin{tcolorbox}
\textbf{US 9}: COMO administrador QUIERO poder definir las jugadas disponibles de los técnicos PARA enriquecer la simulación

\vline

\textbf{Criterios de aceptación:}
\begin{itemize}
\item %CA
\item %CA
\item %CA
\end{itemize}
\end{tcolorbox}
\vspace{10pt}


\begin{tcolorbox}
\textbf{US 10}: COMO participante QUIERO poder crear y aceptar desafíos PARA medirme con otros participantes

\vline

\textbf{Criterios de aceptación:}
\begin{itemize}
\item %CA
\item %CA
\item %CA
\end{itemize}
\end{tcolorbox}
\vspace{10pt}


\begin{tcolorbox}
\textbf{US 11}: COMO participante QUIERO apostar fichas PARA subir posiciones en la tabla

\vline

\textbf{Criterios de aceptación:}
\begin{itemize}
\item %CA
\item %CA
\item %CA
\end{itemize}
\end{tcolorbox}
\vspace{10pt}


\begin{tcolorbox}
\textbf{US 12}: COMO administrador QUIERO que los participantes puedan simular partidas PARA que jueguen entre sí

\vline

\textbf{Criterios de aceptación:}
\begin{itemize}
\item %CA
\item %CA
\item %CA
\end{itemize}
\end{tcolorbox}
\vspace{10pt}


\begin{tcolorbox}
\textbf{US 13}: COMO administrador QUIERO poder ajustar la duración (en turnos) de las simulaciones PARA que el sistema sea flexible

\vline

\textbf{Criterios de aceptación:}
\begin{itemize}
\item %CA
\item %CA
\item %CA
\end{itemize}
\end{tcolorbox}
\vspace{10pt}

\begin{tcolorbox}
\textbf{US 14}: COMO administrador QUIERO poder modificar las fórmulas de resolución de acciones PARA ir ajustando el sistema a lo largo del tiempo

\vline

\textbf{Criterios de aceptación:}
\begin{itemize}
\item %CA
\item %CA
\item %CA
\end{itemize}
\end{tcolorbox}
\vspace{10pt}


\begin{tcolorbox}
\textbf{US 15}: COMO administrador QUIERO que el primer turno de cada simulación sea al azar PARA hacerlo justo

\vline

\textbf{Criterios de aceptación:}
\begin{itemize}
\item %CA
\item %CA
\item %CA
\end{itemize}
\end{tcolorbox}
\vspace{10pt}

\begin{tcolorbox}
\textbf{US 16}: COMO administrador QUIERO que quede un log y toda la información pertinente de cada simulación PARA que cualquier participante pueda consultarlo

\vline

\textbf{Criterios de aceptación:}
\begin{itemize}
\item %CA
\item %CA
\item %CA
\end{itemize}
\end{tcolorbox}
\vspace{10pt}

\begin{tcolorbox}
\textbf{US 17}: COMO administrador QUIERO que se repartan las fichas adecuadamente después de terminado el partido PARA calcular la nueva tabla de posiciones

\vline

\textbf{Criterios de aceptación:}
\begin{itemize}
\item %CA
\item %CA
\item %CA
\end{itemize}
\end{tcolorbox}
\vspace{10pt}

\begin{tcolorbox}
\textbf{US 18}: COMO participante QUIERO ver la tabla de posiciones PARA compararme con los otros jugadores

\vline

\textbf{Criterios de aceptación:}
\begin{itemize}
\item %CA
\item %CA
\item %CA
\end{itemize}
\end{tcolorbox}
\vspace{10pt}

\begin{tcolorbox}
\textbf{US 19}: COMO administrador QUIERO que el presupuesto de cada equipo no supere el cap del participante PARA equilibrar los valores de los equipos.

\vline

\textbf{Criterios de aceptación:}
\begin{itemize}
\item %CA
\item %CA
\item %CA
\end{itemize}
\end{tcolorbox}
\vspace{10pt}


\subsubsection*{Discusiones}
Algunas \textbf{User Stories} en las que hubo discrepancias extremas en la valuación del \textbf{Effort} fueron la \emph{US 1}, \emph{US 2}, y en cuanto al \textbf{Business Value} la \emph{US 2}, \emph{4}, \emph{US 17}.

En el caso del \textbf{Effort} para la \emph{US 1}, tres integrantes del grupo habían puesto un 3, y el restante un 13. Quien le dio más esfuerzo especificó los detalles que involucraría el potencial registro de los usuarios (modelar; definir datos necesarios; formularios de registro, ingreso, recuperación de contraseña y todas las validaciones asociadas; seguridad; etc.), por lo que decidimos ir a mitad de camino y asignarle un 8.

En el caso del \textbf{Effort} para la \emph{US 2}, una situación similar pero más dispersa, esfuerzos empezando por 3 y llegando hasta 13. La valuación más baja se debía a que no se consideraba ``ni un ABM'' a la sección de armado de equipo. Pero, nuevamente, el extremo más alto argumentó que si bien no se hacían modificaciones a los datos, había que tener en cuenta detalles importantes (interfaz de usuario \emph{trabajada} y fácilmente usable, conexión a base de datos para jugadores y técnicos), permitir manipular la visualización de la información (filtros y ordenamiento en base a estadísticas, nombre, etc.), y realizar validaciones (jugadores repetidos, cap de equipo del participante superado), por lo que se pactó un punto medio nuevamente con un 8.

Con respecto al \textbf{Business Value} de \emph{US 2} y \emph{US 4}, se transformó en una especie de dilema del ``huevo y la gallina''. La idea principal del sistema es que los participantes puedan armar su equipo para realizar simulaciones; sin embargo, sin jugadores no pueden armar el equipo, entonces, ¿qué era lo más importante para el negocio? Se terminó decidiendo que armar el equipo era más importante, pero esta relación intensa entre ambos hechos hizo que aumentara el \textbf{Business Value} de \emph{US 4}, que en un principio se le había dado un puntaje más bajo.

En lo que respecta al \textbf{Business Value} de \emph{US 17}, tres integrantes le habían otorgado un 8, y uno un 2. El integrante que asignó el menor puntaje esgrimía que no le añadía valor al negocio, ya que se podía sacar el log del sistema y todo seguía teniendo sentido, las simulaciones se podían hacer igual y el juego se podía jugar al 100\%. Se le explicó, no obstante, que para un participante puede tener mucho valor saber en qué puntos específicos del juego su equipo estuvo fallando para saber cómo mejorar, además de que a los verdaderos fanáticos de este tipo de juegos les encanta ver el paso a paso y tener la mayor cantidad de información posible. Dicho integrante subió el valor a un 5, y siendo tres contra uno y una brecha más corta entre las dos valuaciones, se terminó decidiendo poner un 8.