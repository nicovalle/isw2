\section{Introducción}
\indent \indent Esta primera etapa del trabajo consistió en la planificación, utilizando la metodología ágil Scrum, del desarrollo de un simulador de partidos de básquet de fantasía.

Por las características del sistema pedido, donde hay una simulación -que lleva una porción importante de cómputo y representa una etapa del desarrollo bastante \emph{pesada}-, quedaron pocas user stories en total, pues gran parte del sistema se concentra específicamente en ese punto.

Asimismo, solo determinamos dos roles en el sistema, pues en principio no parecería haber más agentes que participen e influyan directamente en el mismo. 

\begin{itemize}
\item \textbf{Participante}: es quien se encarga de crear equipos, desafiar a otros participantes y participar de las simulaciones. El \textit{usuario final} del sistema.
\item \textbf{Administrador}: es aquel que actualiza los datos de los jugadores, define las jugadas de cada técnico y las configuraciones de la simulación, tales como la cantidad de turnos de cada una. Es un supervisor del sistema, quien lo regula, el que realiza las acciones para hacerlo más atractivo para los participantes, y más equilibrado.
\end{itemize}
