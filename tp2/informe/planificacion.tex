\section{Plan de Proyecto}
\label{sec:planificacion}
\subsection{Iteraciones}
Nuestra decisión sobre la conformación de las iteraciones se rigió por varios factores: 
\begin{itemize}
\item Se tuvieron en cuenta las necesidades extraídas del QAW con los stakeholders
\item Se trató de agrupar a los casos de uso por su temática, prefiriendo en caso de ser posible juntar dentro de una misma iteración los casos de uso que se relacionan con funcionalidades similares.
\item Otro factor que se tuvo en cuenta a la hora de definir el orden de las iteraciones, fueron los riesgos detectados y analizados en la sección de Análisis de Riesgos (\ref{sec:riesgos}).
\item También se consideró la prioridad de la funcionalidades referidas en los casos de uso, intentando en lo posible desarrollar antes las más importantes para el negocio.
\end{itemize}

La primera iteración quedó expresamente definida por el QAW y el análisis de riesgo, factores a los que se les dio mayor prioridad. 
Del análisis de riesgo, y dado el factor prioritario que tiene la disponibilidad del sistema, se extrajo que la parte más conflictiva ocurre a la hora de transmitir los desafíos (especialmente para los eventos globales), así como al mostrar correctamente las simulaciones.

El siguiente punto en cuanto a nivel de riesgo, fue lo concerniente a la seguridad, integridad y transparencia de la transmisión de los datos de pago y movimientos de dinero, y las transmisiones de datos usadas para definir los resultados de los partidos de la realidad y minuto a minuto de las simulaciones. Decidimos encarar esto en la segunda iteración del proyecto.

Otro foco de preocupación y de vital importancia para el proyecto, es su alcance y monetización. Por esa razón, se decidió trabajar en lo que respecta a las publicidades del sitio, simulaciones y transmisiones (y el uso de los datos de comportamiento de usuarios), además de la regionalización del sistema (con todas las reglas y restricciones que eso conlleva) en la tercer iteración.

El manejo administrativo de las cuentas de los usuarios, y el aspecto \emph{social} (chat, utilización de las opiniones vertidas en redes sociales, etc.), consideradas cuestiones de menor complejidad, se decidieron enfrentar en la cuarta iteración.

Decidimos dedicarnos en la quinta iteración a todo lo referido a los desafíos, su creación, configuración, definición de reglas y de los premios. En la sexta iteración, nos encargaremos de lo que concierne a los rankings de los jugadores globales, regionales, etc. Como los casos de uso de estas dos iteraciones tienen un menor riesgo y complejidad que los anteriores, consideramos que estas dos fases son de construcción, ya que se trabajará bastante en lo implementativo y poco en el análisis/diseño.

\subsubsection{Primera Iteración (Elaboración)}
\begin{itemize}
\item (CU19) Mostrando detalle minuto a minuto de la simulación
\item (CU20) Observando evento global / continental
\item (CU22) Observando la transmisión de un partido de liga
\item (CU2)  Obteniendo datos en tiempo real
\item (CU21) Mostrando simulación gráfica
%\item CU15) Mostrando desarrollo (log o streaming) de desafío} (sistema -> usuario)
\end{itemize}

\subsubsection{Segunda iteración (Elaboración)} 
\begin{itemize}
\item (CU9) Creando cuenta de usuario
\item (CU10) Actualizando datos de medios de pago
\item (CU34) Auditando movimientos de dinero
\item (CU33) Consultando estado de cuenta y movimientos de usuario
\item (CU35) Auditando simulaciones
\item (CU8) Apostando
%\item CU43) Logueando movimientos de dinero}: (depende de lo que diga Javi)
\end{itemize}


\subsubsection{Tercera Iteración (Elaboración)}
\begin{itemize}
\item (CU36) Definiendo restricciones por zona
\item (CU25) Definiendo regiones de la plataforma
\item (CU18) Definiendo reglas de simulación
\item (CU38) Configurando publicidad y ads en transmisiones
\item (CU16) Configurando publicidad en el sitio y simulaciones
\item (CU17) Acceder a datos de preferencia/comportamiento de usuarios
%\item (CU44) Mostrando publicidad en la simulación} - ¿es algo que se hace al generar la simulación,  puede ser luego?
%\item (CU37) Configurando publicidades en simulaciones y en el sitio}:  (representante de empresas (?))
%\item (CU39) Consultando datos de usuario}: (representantes de empresas(?))
\end{itemize}


\subsubsection{Cuarta Iteración (Elaboración)}
\begin{itemize}
\item (CU6) Participando del chat general
\item (CU7) Participando de chat privado
\item (CU12) Consultando cuenta de usuario
\item (CU41) Desactivando cuenta
\item (CU42) Reactivando cuenta
\item (CU23) Recolectando opiniones de redes sociales y chats
\item (CU24) Definiendo impacto de opiniones
%4- consultando datos de jugadores /técnicos /equipos (participante)
\end{itemize}

\subsubsection{Quinta Iteración (Construcción)}
\begin{itemize}
\item (CU1) Eligiendo liga para competir
\item (CU3) Definiendo reglas de desafío
\item (CU4) Creando desafío
\item (CU5) Aceptando desafío
\item (CU14) Consultando estado (cuenta regresiva, participantes, posiciones) del desafío
\item (CU13) Definiendo premios
\end{itemize}

\subsubsection{Sexta Iteración (Construcción)}
\begin{itemize}
\item (CU31) Configurando visibilidad de los desafíos
\item (CU32) Reiniciando el ranking de jugadores
\item (CU28) Consultando dashboard regional o global
\item (CU29) Consultando ranking de jugadores
\item (CU26) Definiendo desafíos interzonales
%6 -  definiendo reglas de puntajes (administrador)
\end{itemize}


\subsection{Primera iteración en detalle}
\label{subsec:primeraiteracion}