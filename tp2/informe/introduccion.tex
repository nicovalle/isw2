\section{Introducción}
Debido al éxito rotundo de nuestra implementación particular del Curry Game, los inversores originales, más otros nuevos interesados, decidieron extender considerablemente la plataforma en cuanto a funcionalidad (con el objetivo de tener mayor alcance), incluyendo entre otras cosas la posibilidad de abarcar otros juegos aparte del básquet, basarse no sólo en simulaciones sino también en datos de partidos en tiempo real -como una liga de fantasía tradicional-, poder realizar simulaciones gráficas a través de motores 2D/3D, apuestas con dinero real, exhibir publicidad, y un largo y redituable etcétera.

El objetivo de este trabajo práctico consistió en la planificación de las etapas de \emph{Elaboración}, \emph{Construcción} y \emph{Transición} siguiendo la metodología UP, asumiendo la etapa de \emph{Incepción} como concluida, y en la construcción de una arquitectura que resolviera el problema propuesto teniendo en cuenta las diversos detalles extraídos del enunciado y del QAW.

Basados tanto en el enunciado presentado como en el QAW provisto por la cátedra, se definió una lista de casos de uso (\refcompleta{sec:casosdeuso}). Ortogonalmente a eso, se realizó un análisis de riesgos (\refcompleta{sec:riesgos}, y en combinación con los casos de uso obtenidos se construyó el Plan de Proyecto (\refcompleta{sec:planificacion}), indicando en el mismo las distintas iteraciones que consideramos necesarias para la concreción del desarrollo solicitado, al igual que los casos de uso que lo componen.

Además, por pedido expreso del enunciado, se hizo énfasis en la descripción de la primera iteración del Plan (\refcompleta{subsec:primeraiteracion}), correspondiente a la etapa de \emph{Elaboración}. Se indican los casos de uso que forman parte de dicha iteración, cómo se descompone cada uno en tareas, la dependencia de las mismas, y la estimación en horas hombre de su concreción.

A continuación, se realizó un diagrama de Gantt que detalla la planificación de la primera iteración, indicando en él, para cada tarea, la asignación de los recursos disponibles (los cuatro integrantes del presente trabajo).

Posteriormente, se realizaron numerosos diagramas que conforman la arquitectura que proponemos para el sistema en cuestión. Además, se comparó esta arquitectura con la diagramada para el primer trabajo práctico de la materia.
