\section{Introducción}
Debido al éxito rotundo de nuestra implementación particular del Curry Game, los inversores originales, más otros nuevos interesados, decidieron extender considerablemente la plataforma en cuanto a funcionalidad (con el objetivo de tener mayor alcance), incluyendo entre otras cosas la posibilidad de abarcar otros juegos aparte del básquet, basarse no sólo en simulaciones sino también en datos de partidos en tiempo real -como una liga de fantasía tradicional-, poder realizar simulaciones gráficas a través de motores 2D/3D, apuestas con dinero real, exhibir publicidad, y un largo y redituable etcétera.

El objetivo de este trabajo práctico consistió en la planificación de las etapas de \emph{Elaboración} y \emph{Construcción} siguiendo la metodología UP, asumiendo la etapa de \emph{Incepción} como concluida. 

Basados tanto en el enunciado presentado como en el QAW provisto por la cátedra, se definió una lista de casos de uso. Ortogonalmente a eso, se realizó un análisis de riesgos, y en combinación con los casos de uso obtenidos se construyó el Plan de Proyecto, indicando en el mismo las distintas iteraciones que consideramos necesarias para la concreción del desarrollo solicitado, al igual que los casos de uso que lo componen.

Además, por pedido expreso del enunciado, se hizo énfasis en la descripción de la primera iteración del Plan, correspondiente a la etapa de \emph{Elaboración}. Para dicha iteración, se indican los casos de uso que forman parte de ella, la descomposición en tareas de cada uno de ellos, la dependencia de las mismas, y la estimación en horas hombre de su concreción.

Finalmente, se realizó un diagrama de Gantt que detalla la planificación de la primera iteración, indicando en él, para cada tarea, la asignación de los recursos disponibles (los cuatro integrantes del presente trabajo).
