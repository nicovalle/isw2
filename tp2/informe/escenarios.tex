\section{Atributos de calidad}

En esta sección se presentan los atributos de calidad identificados en el TP (ordenados según la prioridad definida por los \emph{stakeholders} en el \emph{Quality Attribute Workshop}), junto a sus respectivos escenarios.

% \begin{center}
%   \begin{tabular}{| l | p{10cm} | }
%     \hline
% 	\textbf{Descripción} & UNA DESCRIPCION\\  \hline
% 	\textbf{Fuente} & UNA FUENTE\\  \hline
% 	\textbf{Estímulo} & UN ESTIMULO\\  \hline
% 	\textbf{Entorno} & UN ENTORNO\\  \hline
% 	\textbf{Artefacto} & UN ARTEFACTO\\  \hline
% 	\textbf{Respuesta} & UNA RESPUESTA\\  \hline
% 	\textbf{Medición} & UNA MEDICION\\  \hline
%   \end{tabular}
% \end{center} 

\subsection{Disponibilidad}

\subsubsection*{Motivación}
\begin{itemize}
\item El sistema tiene que estar andando todo el tiempo.
\item Cantidad de usuarios limitada - máxima por servidor
\item Múltiples nodos distribuidos de pocos usuarios cada uno.
\item Los partidos streameados deben darse en excelente calidad sin cortes.
\item Cuando ninguno de los motores gráficos se encuentra disponible, debe usarse el streaming del engine 3D.
\item Debe esforzarse por respetar a los países en los que la legislación vigente no permite directamente que se ingrese al sitio.
\item Debe ser fácil desactivar una cuenta por un tiempo, sin poder reactivarla, tanto para usuarios adictos en recuperación como para usuarios de países en los que no está disponible el sitio.
\end{itemize}

\subsubsection*{Escenarios}
\begin{center}
  \begin{tabular}{| l | p{10cm} | }
    \hline
	\textbf{Descripción} & En caso de saturación de un nodo, deben direccionarse nuevas conexiones a otro nodo.\\  \hline
	\textbf{Fuente} & Usuario\\ \hline
	\textbf{Estímulo} & Se realiza un request al sistema.\\  \hline
	\textbf{Entorno} & Nodo saturado\\  \hline
	\textbf{Artefacto} & Nodo del sistema\\  \hline
	\textbf{Respuesta} & Se selecciona un nuevo nodo regional y se continúa el servicio mediante una conexión con él.\\  \hline
	\textbf{Medición} & El 99.99 \% de las nuevas conexiones se realizan sin inconveniente alguno y de forma transparente al usuario.\\  \hline
  \end{tabular}
\end{center} 


\begin{center}
  \begin{tabular}{| l | p{10cm} | }
    \hline
	\textbf{Descripción} & Los streamings de partidos se realizan con la menor cantidad de cortes posibles.\\  \hline
	\textbf{Fuente} & Usuario\\  \hline
	\textbf{Estímulo} & Solicita ver el streaming de un partido\\  \hline
	\textbf{Entorno} & Normal\\  \hline
	\textbf{Artefacto} & Generador gráfico\\  \hline
	\textbf{Respuesta} & Se realizan los cálculos pertinentes para determinar el ancho de banda del usuario y se comienza el streaming con la calidad determinada.\\  \hline
	\textbf{Medición} & Se produce a lo sumo un corte con duración menor a 5 segundos por transmisión.\\  \hline
  \end{tabular}
\end{center}

\begin{center}
  \begin{tabular}{| l | p{10cm} | }
    \hline
	\textbf{Descripción} & Si los motores gráficos para una simulación no están disponibles, debe realizarse un streaming del engine 3D para la simulación. \\  \hline
	\textbf{Fuente} & Usuario \\  \hline
	\textbf{Estímulo} & Intenta observar una simulación\\  \hline
	\textbf{Entorno} & Motores gráficos del dispositivo no disponibles.\\  \hline
	\textbf{Artefacto} & Generador gráfico\\  \hline
	\textbf{Respuesta} & Ante la detección de la imposibilidad de utilizar los motores gráficos en el dispositivo, se activa el streaming del engine 3D de la simulación.\\  \hline
	\textbf{Medición} & Se trasmite la simulación 3D sin inconvenientes en el 99\% de los casos.\\  \hline
  \end{tabular}
\end{center}  

\begin{center}
  \begin{tabular}{| l | p{10cm} | }
    \hline
	\textbf{Descripción} & Si se detecta que los motores gráficos vuelven a estar disponibles, se los vuelve a utilizar en lugar del streaming de la simulación \\  \hline
	\textbf{Fuente} & Identificador de dispositivo\\  \hline
	\textbf{Estímulo} & Detecta motores gráficos disponibles\\  \hline
	\textbf{Entorno} & Streaming de una simulación\\  \hline
	\textbf{Artefacto} & Generador gráfico\\  \hline
	\textbf{Respuesta} & Se utiliza alguno de los motores gráficos del dispositivo\\  \hline
	\textbf{Medición} & Se corta el streaming de video y en menos de 10 segundos se continúa la simulación con el motor gráfico.\\  \hline
  \end{tabular}
\end{center}  

\begin{center}
  \begin{tabular}{| l | p{10cm} | }
    \hline
	\textbf{Descripción} & Si un usuario idenficado como adicto en recuperación intenta conectarse, se le niega el acceso. \\  \hline
	\textbf{Fuente} & Usuario identificado como adicto\\  \hline
	\textbf{Estímulo} & Intento de conexión\\  \hline
	\textbf{Entorno} & Normal\\  \hline
	\textbf{Artefacto} & Controlador de Alta de Usuarios\\  \hline
	\textbf{Respuesta} & Se le niega el acceso al sistema.\\  \hline
	\textbf{Medición} & En el 99.9999 \% de los casos de intento de conexión proveniente de un usuario identificado como adicto en recuperación, se le niega el acceso al sistema con un cartel informativo.\\  \hline
  \end{tabular}
\end{center}  

\begin{center}
  \begin{tabular}{| l | p{10cm} | }
    \hline
	\textbf{Descripción} & Se bloquea el acceso desde países que prohiben el uso del sistema.\\  \hline
	\textbf{Fuente} & Usuario de país que prohibe el uso del sistema\\  \hline
	\textbf{Estímulo} & Intento de conexión\\  \hline
	\textbf{Entorno} & Normal\\  \hline
	\textbf{Artefacto} & Controlador de Alta de Usuarios\\  \hline
	\textbf{Respuesta} & Se le niega el acceso al sistema.\\  \hline
	\textbf{Medición} & En el 99.9999 \% de los casos de intento de conexión proveniente de un país que no permite el acceso al sitio, se le niega el accesso y se le muestra un cartel informativo.\\  \hline
  \end{tabular}
\end{center} 


\subsection{Seguridad}

\subsubsection*{Motivación}
\begin{itemize}
\item Todo lo relativo al manejo del dinero (depósito y retiro de los participantes vía tarjeta de crédito o caja de ahorro) deber ser seguro y transparente.
\item Se teme por la seguridad de los datos de los usuarios, tanto por el robo de la información de las tarjetas de crédito o cajas de ahorro como de la explotación de los datos de los usuarios por terceros para fines no autorizados.
\item Preocupa el resultado de los desafíos, por el pago/cobro a los participantes, y la coherencia con los datos provenientes de las empresas que relevan los resultados de los partidos.
\item Es importante proveer transparencia acerca del funcionamiento de las simulaciones. 
\item Los módulos de las simulaciones deben poder ser inspeccionados fácilmente por entidades de control.
\item Se debe loggear todo movimiento de dinero para evitar cualquier tipo de evasión impositiva.
\end{itemize}

\subsubsection*{Escenarios}
\begin{center}
  \begin{tabular}{| l | p{10cm} | }
    \hline
	\textbf{Descripción} & La información de los medios de pago de los usuarios está protegida contra el robo de datos\\  \hline
	\textbf{Fuente} & Atacante externo\\  \hline
	\textbf{Estímulo} & Intenta vulnerar la información de medios de pago de los usuarios mediante el descifrado de los datos.\\  \hline
	\textbf{Entorno} & Normal\\  \hline
	\textbf{Artefacto} & Datos del Almacén de información de crédito \\  \hline
	\textbf{Respuesta} & Los datos en texto plano no son accesibles al atacante en tiempos razonables.\\  \hline
	\textbf{Medición} & En el 99.9999 \% de los casos, los ataques no son exitosos.\\  \hline
  \end{tabular}
\end{center}

\begin{center}
  \begin{tabular}{| l | p{10cm} | }
    \hline
	\textbf{Descripción} & Los movimientos de dinero se registran en el sistema\\  \hline
	\textbf{Fuente} & Usuario o Gestor de crédito\\  \hline
	\textbf{Estímulo} & Se apuesta dinero o se recibe un pago.\\  \hline
	\textbf{Entorno} & Normal\\  \hline
	\textbf{Artefacto} & Gestor de crédito\\  \hline
	\textbf{Respuesta} & Se realiza correctamente el movimiento y se loggea un registro con los datos de la transacción\\  \hline
	\textbf{Medición} & El registro se guarda correctamente en el 99.999 \% de los casos.\\  \hline
  \end{tabular}
\end{center}  

\begin{center}
  \begin{tabular}{| l | p{10cm} | }
    \hline
	\textbf{Descripción} & Un usuario no puede consultar los datos de pago que no le corresponden.\\  \hline
	\textbf{Fuente} & Usuario\\  \hline
	\textbf{Estímulo} & Intenta obtener los datos de pago de otro usuario.\\  \hline
	\textbf{Entorno} & Normal\\  \hline
	\textbf{Artefacto} & Gestor de crédito\\  \hline
	\textbf{Respuesta} & Se rechaza el intento y se loggea un registro de la solicitud.\\  \hline
	\textbf{Medición} & La solicitud se rechaza un 99.99 \% de las veces.\\  \hline
  \end{tabular}
\end{center}

\begin{center}
  \begin{tabular}{| l | p{10cm} | }
    \hline
	\textbf{Descripción} & Los resultados de los desafíos se corresponden con los datos provistos por las empresas relevadoras de datos de los partidos\\  \hline
	\textbf{Fuente} & Empresa relevadora de datos de partidos\\  \hline
	\textbf{Estímulo} & Se envían resultados minuto a minuto de un partido\\  \hline
	\textbf{Entorno} & Normal\\  \hline
	\textbf{Artefacto} & Controlador de desafíos\\  \hline
	\textbf{Respuesta} & Se autentica la identidad de la empresa relevadora de datos y se persisten los resultados en el sistema.\\  \hline
	\textbf{Medición} & En más del 99\% de los casos los resultados de los desafíos se contrastan correctamente con los datos provistos por las empresas.\\  \hline
  \end{tabular}
\end{center}

\begin{center}
  \begin{tabular}{| l | p{10cm} | }
    \hline
	\textbf{Descripción} & Se rechazan los datos de los partidos de fuentes no autenticadas correctamente\\  \hline
	\textbf{Fuente} & Atacante externo\\  \hline
	\textbf{Estímulo} & Se hace pasar por una de las empresas contratadas para proveer los datos y de esta forma alterar los resultados de los desafíos\\  \hline
	\textbf{Entorno} & Normal\\  \hline
	\textbf{Artefacto} & Controlador de desafíos\\  \hline
	\textbf{Respuesta} & Se detecta la intrusión al no poder autenticar la identidad de la fuente de datos, se rechazan los datos y se registra el potencial ataque para futuros análisis\\  \hline
	\textbf{Medición} & En más del 99.99\% de los casos el ataque es correctamente detectado y repelido.\\  \hline
  \end{tabular}
\end{center}    


\begin{center}
  \begin{tabular}{| l | p{10cm} | }
    \hline
	\textbf{Descripción} & Un usuario puede acceder a sus datos de medios de pago\\  \hline
	\textbf{Fuente} & Usuario\\  \hline
	\textbf{Estímulo} & Intenta consultar o modificar sus datos de medios de pago\\  \hline
	\textbf{Entorno} & Normal\\  \hline
	\textbf{Artefacto} & Controlador de Alta de Medios de pago\\  \hline
	\textbf{Respuesta} & Se autoriza el acceso.\\  \hline
	\textbf{Medición} & La solicitud se autoriza 99.9999 \% de las veces.\\  \hline
  \end{tabular}
\end{center} 


%%% REPETIDO!
%\begin{center}
%  \begin{tabular}{| l | p{10cm} | }
%    \hline
%	\textbf{Descripción} & Un usuario puede acceder a sus datos de medios de pago\\  \hline
%	\textbf{Fuente} & Usuario\\  \hline
%	\textbf{Estímulo} & Intenta consultar o modificar sus datos de medios de pago\\  \hline
%	\textbf{Entorno} & Normal\\  \hline
%	\textbf{Artefacto} & Controlador de Medios de pago\\  \hline
%	\textbf{Respuesta} & Se autoriza el acceso.\\  \hline
%	\textbf{Medición} & La solicitud se autoriza 99.99 \% de las veces.\\  \hline
%  \end{tabular}
%\end{center}


\begin{center}
  \begin{tabular}{| l | p{10cm} | }
    \hline
	\textbf{Descripción} & El funcionamiento de los módulos de simulación es el esperado\\  \hline
	\textbf{Fuente} & Auditor\\  \hline
	\textbf{Estímulo} & Prueba los módulos en servidores testigo\\  \hline
	\textbf{Entorno} & Entorno de prueba\\  \hline
	\textbf{Artefacto} & Calculador de jugadas\\  \hline
	\textbf{Respuesta} & Se ejecutan los casos de prueba de los sistemas testigos.\\  \hline
	\textbf{Medición} & Se superan todos los casos de prueba de los sistemas testigos.\\  \hline
  \end{tabular}
\end{center}

\begin{center}
  \begin{tabular}{| l | p{10cm} | }
    \hline
	\textbf{Descripción} & El software instalado en los servidores del sistema se corresponde con el realizado por el equipo de desarrollo.\\  \hline
	\textbf{Fuente} & Auditor y equipo de desarrollo\\  \hline
	\textbf{Estímulo} & Se ejecutan mecanismos de hashing para comprar los componentes.\\  \hline
	\textbf{Entorno} & Entorno de prueba\\  \hline
	\textbf{Artefacto} & Sistema\\  \hline
	\textbf{Respuesta} & Se obtiene el hash del software instalado en los servidores.\\  \hline
	\textbf{Medición} & En todos los casos el hash obtenido es igual al hash esperado por el equipo de desarrollo.\\  \hline
  \end{tabular}
\end{center}  




\subsection{Modificabilidad}

\subsubsection*{Motivación}
\begin{itemize}
\item Se deben poder agregar fácilmente nuevos deportes
\item Debe resultar fácil modificar las simulaciones con el fin de mejorarlas.
\item Los datos del sitio deben ser fácilmente minados, y además debe poder crearse fácilmente reportes a partir de ellos.
\item Se deben poder modificar fácilmente las publicidades, tanto de las simulaciones como del sistema.
\end{itemize}

\subsubsection*{Escenarios}
\begin{center}
  \begin{tabular}{| l | p{10cm} | }
    \hline
	\textbf{Descripción} & Es fácil agregar un nuevo deporte al sistema.\\  \hline
	\textbf{Fuente} & Desarrollador\\  \hline
	\textbf{Estímulo} & Introduce un nuevo deporte al sistema.\\  \hline
	\textbf{Entorno} & Tiempo de desarrollo y diseño\\  \hline
	\textbf{Artefacto} & Sistema\\  \hline
	\textbf{Respuesta} & Los cambios se agregan sin ninguna complicación y se pueden realizar simulaciones del nuevo deporte.\\  \hline
	\textbf{Medición} & En menos de 40 horas hombre debe poder realizarse la integración del nuevo deporte al sistema sin afectar negativamente a las funcionalidades previamente provistas. El grueso del costo temporal se dedica al modelado y relevado del deporte en cuestión y no a la integración en el sistema.\\  \hline
  \end{tabular}
\end{center}

\begin{center}
  \begin{tabular}{| l | p{10cm} | }
    \hline
	\textbf{Descripción} & Se pueden modificar fácilmente las simulaciones para mejorarlas.\\  \hline
	\textbf{Fuente} & Desarrollador\\  \hline
	\textbf{Estímulo} & Introduce cambios en el módulo de simulación.\\  \hline
	\textbf{Entorno} & Tiempo de desarrollo y diseño\\  \hline
	\textbf{Artefacto} & Sistema\\  \hline
	\textbf{Respuesta} & Los cambios se agregan sin ninguna complicación, y la nueva versión del simulador inmediatamente se pone en marcha.\\  \hline
	\textbf{Medición} & En menos de 50 horas hombre se pueden agregar cambios a los módulos de simulación.\\  \hline
  \end{tabular}
\end{center}

\begin{center}
  \begin{tabular}{| l | p{10cm} | }
    \hline
	\textbf{Descripción} & Se pueden agregar nuevos reportes de manera fácil y rápida.\\  \hline
	\textbf{Fuente} & Desarrollador\\  \hline
	\textbf{Estímulo} & Intorudce un nuevo reporte.\\  \hline
	\textbf{Entorno} & Tiempo de desarrollo y diseño\\  \hline
	\textbf{Artefacto} & Sistema\\  \hline
	\textbf{Respuesta} & Se realiza el reporte y se lo agrega al sistema sin ningún incoveniente.\\  \hline
	\textbf{Medición} & En menos de 30 horas hombres se puede generar un nuevo reporte con los datos persistidos en el sistema.\\  \hline
  \end{tabular}
\end{center}

\subsection{Performance}

\subsubsection*{Motivación}
\begin{itemize}
\item Se requiere que las transmisiones de los partidos se realicen con excelente calidad y sin cortes.
\item Las simulaciones deben realizarse en tiempo real minuto a minuto.
\item Los eventos globales deben funcionar correctamente.
\end{itemize}

\subsubsection*{Escenarios}
\begin{center}
  \begin{tabular}{| l | p{10cm} | }
    \hline
	\textbf{Descripción} & En lo posible las simulaciones se grafican en los dispositivos del usuario y no en el sistema.\\ \hline
	\textbf{Fuente} & Usuario\\  \hline
	\textbf{Estímulo} & Visualiza una simulación.\\  \hline
	\textbf{Entorno} & Normal\\  \hline
	\textbf{Artefacto} & Graficador de simulaciones\\  \hline
	\textbf{Respuesta} & Se envían solamente los datos necesarios para que el engine opere en el dispositivo del cliente.\\ \hline
	\textbf{Medición} & En 9 de cada 10 dispositivos se grafica la simulación en alguno de los dos motores gráficos en vez de streamear el engine 3D.\\  \hline
  \end{tabular}
\end{center} 

\begin{center}
  \begin{tabular}{| l | p{10cm} | }
    \hline
	\textbf{Descripción} & Se puede consumir la transmisión de un partido de manera rápida y con buena calidad.\\  \hline
	\textbf{Fuente} & Usuario\\  \hline
	\textbf{Estímulo} & Visualiza un partido y se detecta transmisión no óptima.\\  \hline
	\textbf{Entorno} & Ancho de banda del usuario limitado\\  \hline
	\textbf{Artefacto} & Streameador de videos\\  \hline
	\textbf{Respuesta} & Se ejecuta los calculos pertinentes para acomodar el bitrate del video de manera tal que se ajuste acordemente al ancho de banda disponible para el usuario\\  \hline
	\textbf{Medición} & En el 99 \% de los casos, el usuario puede visualizar un partido de manera fluida esperando. El tiempo de espera para el comienzo del streaming de video es menor a 20 segundos.\\  \hline
  \end{tabular}
\end{center}

\begin{center}
  \begin{tabular}{| l | p{10cm} | }
    \hline
	\textbf{Descripción} & Los eventos globales se visualizan correctamente\\  \hline
	\textbf{Fuente} & Usuario\\  \hline
	\textbf{Estímulo} & Visualización de un evento global\\  \hline
	\textbf{Entorno} & Sistema con altos índices de demanda para la visualización del evento.\\  \hline
	\textbf{Artefacto} & Streameador de videos\\  \hline
	\textbf{Respuesta} & De acuerdo a los anchos de banda de los usuarios se transmite el evento de la mejor manera posible.\\  \hline
	\textbf{Medición} & Más del 99 \% de los usuarios pueden visualizar el evento correctamente sin cortes y con tiempos de espera menores al minuto para comenzar la visualización.\\  \hline
  \end{tabular}
\end{center}    


\subsection{Usabilidad}

\subsubsection*{Motivación}
\begin{itemize}
\item{Debe resultar simple y rápido crear equipos y participar de desafíos}
\item{Los administradores deben poder acceder a un dashboard en tiempo real de estado de cuenta regional del sitio y de los grupos de usuarios y analizarlo fácilmente}
\end{itemize}

\subsubsection*{Escenarios}
\begin{center}
  \begin{tabular}{| l | p{10cm} | }
    \hline
	\textbf{Descripción} & Es fácil y rápido crear un equipo.\\  \hline
	\textbf{Fuente} & Usuario\\  \hline
	\textbf{Estímulo} & Intenta crear un nuevo equipo.\\  \hline
	\textbf{Entorno} & Normal\\  \hline
	\textbf{Artefacto} & Interfaz de Usuario\\  \hline
	\textbf{Respuesta} & Se crea correctamente un equipo.\\  \hline
	\textbf{Medición} & En un tiempo promedio de 20 minutos el usuario puede crear un equipo a su parecer.\\  \hline
  \end{tabular}
\end{center}


\begin{center}
  \begin{tabular}{| l | p{10cm} | }
    \hline
	\textbf{Descripción} & Un administrador puede acceder al dashboard de estado de cuenta y de los grupos de participantes de manera rápida y simple.\\  \hline
	\textbf{Fuente} & Administrador\\  \hline
	\textbf{Estímulo} & Accede al dashboard\\  \hline
	\textbf{Entorno} & Normal\\  \hline
	\textbf{Artefacto} & Interfaz de Administrador\\  \hline
	\textbf{Respuesta} & Se muestran el dashboard y sus opciones\\  \hline
	\textbf{Medición} & En menos de de un minuto el administrador puede consultar la información que requiere del dashboard.\\  \hline
  \end{tabular}
\end{center}