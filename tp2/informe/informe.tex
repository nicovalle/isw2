\documentclass[10pt, a4paper,english,spanish]{article}

% \usepackage{a4wide}
\parindent = 0 pt
\parskip = 11 pt
\usepackage[width=15.5cm, left=3cm, top=2.5cm, height= 24.5cm]{geometry}
%Margenes de la pagina.  otra opcion, usar \usepackage{a4wide}
%\usepackage[paper=a4paper, left=0.8cm, right=0.8cm, bottom=1.3cm, top=0.9cm]{geometry}
\usepackage{color}
\usepackage{amsmath}
\usepackage{amsfonts}
%este paquete permitcodebe incluir acentos.  Notar que espera un formato ANSI-blah de archivo.  Si en lugar de eso se tiene un utf8 (usual en los linux), entonces usar \usepackage[utf8]{inputenc}
\usepackage[utf8]{inputenc}

\usepackage{tcolorbox}

%Este paquete es para que algunos titulos (como Tabla de Contenidos) esten en castellano
\usepackage[spanish]{babel}

%El siguiente paquete permite escribir la caratula facilmente
\usepackage{caratula}

\usepackage{framed}

\usepackage{graphicx}
\usepackage{float}

\usepackage{algorithm}
\usepackage{algorithmic}
\usepackage{hyperref}

%\usepackage{algpseudocode}

\newcommand{\real}{\mathbb{R}}
\newcommand{\nat}{\mathbb{N}}

\newcommand{\revJ}[1]{{\color{red} #1}}

%Datos para la caratula
\materia{Ingeniería de Software II}

\titulo{Trabajo Práctico II}
\subtitulo{Tutor: Javier Martínez Viademonte}

\integrante{Laporte, Matías}{686/09}{matiaslaporte@gmail.com}
\integrante{Salegas, Matías}{750/01}{matias.salegas@gmail.com}
\integrante{Vallejo, Nicolás}{500/10}{nico\_pr08@hotmail.com}
\integrante{Zanitti, Gastón}{58/10}{gzanitti@gmail.com}

\begin{document}

%esto construye la caratula
\maketitle 

\tableofcontents
  \newpage

\section{Introducción}
\indent \indent Esta primera etapa del trabajo consistió en la planificación, utilizando la metodología ágil Scrum, del desarrollo de un simulador de partidos de básquet de fantasía.

\subsection*{User Stories}
Por las características del sistema pedido, donde hay una simulación -que lleva una porción importante de cómputo y representa una etapa del desarrollo bastante \emph{pesada}-, quedaron pocas user stories en total, pues gran parte del sistema se concentra específicamente en ese punto. Esto fue algo que se habló con el tutor y estaba dentro de lo esperable.

Justamente por ser algo que llevaba tanto tiempo, se estuvo en la duda de si convenía o no dividir la \textbf{User Story} correspondiente a la simulación. Una solución propuesta fue especificar en las user stories que una simulación se componía de turnos, éstos de jugadas, y éstas de acciones de los jugadores. Una división de ese modo resultó exagerada, y además iría en contra del principio de independencia para las user stories de \textbf{Scrum}; dado que la simulación debería entrar completa en un único sprint, por lo tanto, se decidió dejarla como una única \textbf{User Story}.


\subsubsection*{Valuación User Stories}
Tanto para la sección de \emph{Business Value} como de \emph{Effort} de cada User Story se decidió realizar poker planning entre los 4 integrantes del grupo. Cuando había discrepancias, se esgrimían los argumentos por los que cada uno había puesto el puntaje correspondiente, de manera de intentar convencer a los otros y así converger los criterios.

\subsection*{Roles}
Otro punto donde hubo ciertas dudas acerca de si se estaba o no en uno de los muchísimos caminos correctos posibles, fue en cuanto a los roles. A primera vista, no parecería haber nadie más que participe del sistema más que el usuario final, a quien llamamos un \emph{participante}. 

Leyendo con un poco más de atención y por cómo se encontraban redactados algunos puntos específicos del enunciado, dejando algunas cosas abiertas con la posibilidad de que sufran modificaciones a futuro, nos pareció propicio considerar un rol de alguien que se encarga de ``mantener'' y administrar el sistema -que seguramente no sea el \emph{dueño}, aunque sí puede que esté dirigido por el mismo-. Ese sería el rol del \emph{administrador}.


\begin{itemize}
\item \textbf{Participante}: es quien se encarga de crear equipos, desafiar a otros participantes y participar de las simulaciones. El \textit{usuario final} del sistema.
\item \textbf{Administrador}: es aquel que actualiza los datos de los jugadores, define las jugadas de cada técnico y las configuraciones de la simulación, tales como la cantidad de turnos de cada una. Es un supervisor del sistema, quien lo regula, el que realiza las acciones para hacerlo más atractivo para los participantes, y más equilibrado.
\end{itemize}

  \newpage
\section{Casos de Uso}
\label{sec:casosdeuso}
A continuación se presenta una lista de los casos de uso que el grupo identificó a partir del enunciado y del QAW provisto. Intentan ser lo suficientemente exhaustivos para cubrir la funcionalidad requerida del sistema a desarrollar.

En conjunto con el análisis de riesgos presentado en una sección subsiguiente, la lista presentada se utilizó para definir el alcance de las iteraciones del plan de proyecto, así como para determinar con mayor nivel de detalle la primer iteración del plan.

Los casos de uso están agrupados aproximadamente según funcionalidad; dado que algunos a veces abarcan más de una de las funcionalidades con las que se los decidió jerarquizarlos a gran escala, realizar una categorización estricta resultaba complejo.


%%%% OBSOLECTOS
%** 40- *consultando datos de jugadores /tecnicos/equipos (participante)
%\item (CU43) - * \textbf{Logueando movimientos de dinero}: (depende de lo que diga Javi) \textbf{---- ??????? -------}
%\item (CU15) -  \textbf{Mostrando desarrollo (log o streaming) de desafío} (sistema -> usuario)
%\item (CU44) - ? \textbf{Mostrando publicidad en la simulación} - ¿es algo que se hace al generar la simulación,  puede ser luego?
%\item (CU37) .- \textbf{Configurando publicidades en simulaciones y en el sitio}:  (representante de empresas (?))
%\item (CU39) - * \textbf{Consultando datos de usuario}: (representantes de empresas(?))
%\item (CU27) - *\textbf{Definiendo reglas de puntajes}:  (administrador)


\subsection{Desafíos}
\subsubsection{Administración de desafíos}
\begin{itemize}
\item (CU3) - 1? \textbf{Definiendo reglas de desafío}: un administrador define las reglas de los partidos de cada liga en base al desempeño de los jugadores reales (cantidad de puntos en el desafío, según acción en la realidad).
\item (CU4) - A2? \textbf{Creando desafío}: un administrador o un participante crea un desafío en el que pueden anotarse una cantidad libre de jugadores.
\item (CU13) - 3? \textbf{Definiendo premios}: el creador de un desafío (un administrador o un participante) elige cómo distribuir los premios o porcentajes del dinero apostado, según las posiciones finales del mismo.
\item (CU26) - * \textbf{Definiendo desafíos interzonales}: un administrador del sistema define desafíos para que participen los mejores jugadores de cada una de las regiones definidas.
\item (CU31) - * \textbf{Configurando visibilidad de los desafíos}: el administrador configura la visibilidad de los desafíos con el fin de que participantes no calificados para participar puedan visualizarlos.
\end{itemize}

\subsubsection{Participación en desafíos}

\begin{itemize}
\item (CU1) - A* \textbf{Eligiendo liga para competir}: un participante elige en qué deporte y liga quiere inscribir a un equipo.
\item (CU5) - A* \textbf{Aceptando desafío}: un participante se anota para participar en uno de los desafíos ya creados.
\item (CU14) - \textbf{Consultando estado (cuenta regresiva, participantes, posiciones) del desafío}: un participante puede consultar datos del desafío donde decidió involucrarse: cuánto falta para que empiece, datos de apuestas, quiénes son los otros participantes, sus posiciones, etc.
\end{itemize}



\subsection{Monetización}
\subsubsection{Dinero real}
\begin{itemize}
\item (CU8) - A* \textbf{Apostando}: un participante apuesta cierta cantidad de dinero en un desafío.
\item (CU10) \textbf{Actualizando datos de medios de pago}: el participante asocia algún medio de pago para poder apostar en los desafíos.
\item (CU33) - * \textbf{Consultando estado de cuenta y movimientos de usuario}: un usuario consulta su estado de cuenta y movimiento. Un administrador puede consultar los datos de cualquier participante del sistema.
\item (CU34) - A* \textbf{Auditando movimientos de dinero}: una entidad gubernamental de control puede auditar los movimientos de todos los usuarios para analizar el movimiento de dinero en el sistema.
\end{itemize}

\subsubsection{Publicidad y marketing}
\begin{itemize}
\item (CU16) - * \textbf{Configurando publicidad en el sitio y simulaciones}: un representante de empresa sponsor del proyecto puede acceder a una interfaz desde donde se configura la publicidad que se muestra en el sitio y las simulaciones a los participantes.
\item (CU38) -* \textbf{Configurando publicidad y ads en transmisiones}: un representante de empresa dueña de los derechos de televisación puede modificar las publicidades y ads que se muestran en las transmisiones de los partidos a los participantes.
\item (CU17) - \textbf{Acceder a datos de preferencia/comportamiento de usuarios}: un administrador o un representante de empresa accede a estadísticas que permiten obtener insights de negocio, en base al comportamiento y las opciones más utilizadas por los usuarios.
\end{itemize}

\subsection{Registro, Autenticación y Datos de usuario}

\subsubsection{Cuentas de usuarios}
\begin{itemize}
\item (CU9) - * \textbf{Creando cuenta de usuario}: un participante se registra en el sistema para poder participar de los desafíos
\item (CU12)- * \textbf{Consultando cuenta de usuario}: un participante o administrador consulta los datos con los que se dio de alta el usuario en el sistema.
\item (CU41) - *\textbf{Desactivando cuenta}: un participante o representante de una organización de protección al consumidor puede desactivar una cuenta temporalmente para ayudar a adictos al juego a no tener recaídas.
\item (CU42) - *\textbf{Reactivando cuenta}: un participante puede reactivar su cuenta luego del tiempo establecido en la desactivación.
\item (CU28) - * \textbf{Consultando dashboard regional o global}: un administrador o representante de empresa sponsor del proyecto puede acceder a un dashboard con el estado de cuenta en tiempo real del sitio para cada una de las regiones y niveles y de cualquier grupo de participantes.
\end{itemize}

\subsubsection{Ranking y puntajes}

\begin{itemize}
\item (CU29) -* \textbf{Consultando ranking de jugadores}: un participante puede consultar el ranking de jugadores en cualquier momento y ver su posición en el mismo.
\item (CU32) - * \textbf{Reiniciando el ranking de jugadores}: un administrador reinicia el ranking actual de jugadores, volviendo a todos a la situacion inicial para dar inicio a una nueva temporada.
\end{itemize}


\subsection{Social}
\begin{itemize}
\item (CU6) - * \textbf{Participando del chat general}: un participante envía un mensaje al chat general para que lo vean todos los demás participantes del desafío.
\item (CU7) -.* \textbf{Participando de chat privado}: un participante envía un mensaje privado a otro participante y solo éste puede verlo.
\item (CU23)- ? \textbf{Recolectando opiniones de redes sociales y chats}: se recolectan opiniones de las redes sociales y los chats generales y privados del sistema para afectar los resultados de las simulaciones.
\item (CU24)- * \textbf{Definiendo impacto de opiniones}: un administrador define de qué forma los comentarios obtenidos en las redes sociales impactan en el resultado de los desafíos o en la performance de los jugadores.
\end{itemize}

\subsection{Simulaciones}
\subsubsection{Ejecución}
\begin{itemize}
\item (CU18) - \textbf{Definiendo reglas de simulación}: un administrador modifica las reglas de las simulaciones de los deportes basándose en los comentarios recibidos por el comité de expertos de cada uno de ellos.
\item (CU2) - * \textbf{Obteniendo datos en tiempo real}: se obtienen en tiempo real, mediante empresas proveedoras, datos de los jugadores y del desarrollo de los partidos de las APIs provistas para mantenerlos actualizados. 
\end{itemize}


\subsubsection{Visualización}
\begin{itemize}
\item (CU19) - ? \textbf{Mostrando detalle minuto a minuto de la simulación}: los participantes pueden ver un detalle minuto a minuto de las simulaciones.
\item (CU20) - * \textbf{Observando evento global / continental}: un participante puede acceder a desafíos donde no puede participar en modo espectador, según las reglas dispuestas por los administradores.
\item (CU21) - * \textbf{Mostrando simulación gráfica (motor 2d, motor 3d)}: se genera una simulación gráfica que se muestra al usuario para que pueda disfrutar del desafío de otra manera.
\item (CU22) - * \textbf{Observando la transmisión de un partido de liga}: los participantes pueden ver la transmisión de los partidos en los desafíos en modo \emph{liga de fantasía tradicional}.
\item (CU35) - A* \textbf{Auditando simulaciones}: una entidad gubernamental de control puede auditar las simulaciones para corroborar que se correspondan con los resultados obtenidos y los desafíos pagados.
\end{itemize}


\subsection{Regionalización}
\begin{itemize}
\item (CU25) - * \textbf{Definiendo regiones de la plataforma}: un administrador de la plataforma define regiones y niveles para los mismos, con el fin de regionalizar la plataforma y facilitar la comunicación entre los participantes, la integración con los bancos, el cumplimiento de legislación vigente en los distintos países, etc.
\item (CU36) - * \textbf{Definiendo restricciones por zona}: un administrador define qué deportes están disponibles para cada una de las zonas y si es posible apostar con dinero real, etc. Involucra también definir si el sitio es accesible en cada país.
\end{itemize}







  \newpage
\section{Riesgos}
Análisis de Riesgos

Riesgo 1

Descripción: El experto en redes de la empresa proveedora de infraestructura exige de manera prioritario para el próximo mes el diagrama híbrido  c&c/allocation>deployment
que detalla el manejo del streaming de partidos/simulaciones a todas las regiones en el caso de un evento global o continental. Dado que los mientras del equipo carecen del conocimiento necesario en el tema, es posible que no se llegue a cumplir el plazo.
Probabilidad: Alta
Impacto: Alto
Exposición: Alta
Mitigación : Priorizar el desarrollo del mecanismo de streaming, introduciéndolo en la primera iteración.
Plan de Contingencia : Contratar los servicios de una empresa especializada en el tema
que se encargue del desarrollo.

Riesgo 2
Descripción : Los datos correspondientes a los medios de pago, es decir, tarjetas de crédito
y números de cuentas bancarias, deben almacenarse de forma segura y confidencial  a modo de evitar ataques destinados al robo de información. Dado la sensibilidad de la información no sería totalmente inesperado ser objeto de un ataque.
Probabilidad : Baja
Impacto: Alto
Exposición : Media
Mitigación : Encriptar los datos de los medios de pago de los participantes y de los canales de comunicación de la plataforma. Además, agregar mecanismos verificación de la identidad de los participantes:
Plan de Contingencia: En caso de detección de ataque, aislar los servidores donde se guardan los datos, desconectándolos del sistema, hasta tanto no se haya resuelto la intromisión.


Riesgo 3
Descripción : Es importante que el sistema esté disponible la mayor cantidad del tiempo posible.Hay que tener especial cuidado ya que se rumorea que los servidores de la empresa proveedora de servicios se caen sin previo aviso, y la caída del sistema implicaría
pérdida de dinero.
Probabilidad: Media
Impacto: Alto
Exposición: Alta
Mitigación: Definir un mecanismo de redundancia de servidores. Cualquier acción como la carga de datos por parte del usuario podría hacerse de forma offline y sincronizarse con el sistema una vez resuelto el problema de conexión.
Plan de Contingencia: Contratar otra empresa proveedora de servicios, ya sea como respaldo o reemplazando a la actual como proveedora principal.


Riesgo 4: 
Descripción: La trasmisión de la partidos debe hacerse con la mejor calidad posible y sin cortes ya que no hacerlo impediría cumplir con el estándar de calidad que exigen las ligas para mantener los derechos de transmisión.
Probabilidad: Media
Impacto : Alto
Exposición: Alta
Mitigación: Desarrollar un mecanismo de bitrate variable para el streaming de video, dependiente de las características de la conexión de los usuarios.
Plan de Contingencia: Tercerizar el streaming a través de una plataforma con harta experiencia en el tema, como podría ser Youtube.


Riesgo 5

Descripción: Dado el método de definición de las simulaciones y de los desafíos, es de vital
importancia que los datos provenientes de las empresas proveedoras de los resultados minuto a minuto de los encuentros sean totalmente confiables y representaciones correctas de los desarrollos de los encuentros de la realidad. La inconsistencia de estos datos puede traer como consecuencia resultados incorrectos en los desafíos y el pago erróneo de premios.
Probabilidad: Baja
Impacto : Alto
Exposición: Media
Mitigación: Diseñar un mecanismo de transmisión de los datos de las empresas proveedoras de los resultados minuto a minuto que sea confiable, íntegro y seguro.
Además, desarrollar un algoritmo que se base en contrastar los resultados provistos por más de una empresa para el mismo encuentro con el fin de detectar inconsistencias.
Plan de Contingencia: Deshabilitar el minuto a minuto en el momento en el que se desconfíe de los datos, hasta tanto no se solucione el tema. En tal situación, los resultados de los desafíos se definirán basados en los datos que se puedan obtener luego de los encuentros,en cuyo caso pueden validarse de manera más fácil.


Riesgo 6

Descripción: Es importante proveer el acceso correspondiente tanto a la plataforma en su totalidad como a sus diferentes secciones de acuerdo a la legislación vigente de cada país. No hacerlo podría incurrir en la comisión de delitos.
Probabilidad: Media
Impacto: Alto
Exposición: Alta
Mitigación : Diseñar e implementar un mecanismo lo suficientemente elástico para configurar el acceso a las distintas modalidades de la plataforma. Mantener un equipo que periódicamente se informa acerca de las distintas legislaciones de cada país.
Plan de Contingencia: En caso de poseer dudas con respecto a la legalidad del funcionamiento de la plataforma en un país, impedir el ingreso al sitio hasta tanto se haya validado y consultado sobre las normativas, evitando así potenciales actos ilegales.


Riesgo 7
Descripción : Es menester mantener la transparencia de los módulos de simulaciones, a modo de evitar la interferencia de agentes externos en su ejecución. Un ataque al sistema podría involucrar la alteración del módulo de simulación, provocando la existencia de resultados adulterados.
Probabilidad: Baja
Impacto : Medio
Exposición: Baja
Mitigación: Desarrollar mecanismos de testeo de la simulación en servidores testigos donde los resultados se sepan de antemano. Diseñar un mecanismo que ante inconsistencias realice un deploy de una versión válida del módulo simulador.
Plan de Contingencia: Prohibir la realización de simulaciones hasta tanto se verifique el correcto funcionamiento.


Riesgo 8
Descripción: Los stakeholders quieren tener confianza en que vamos a ser capaces de guardar los datos de tarjetas de crédito, como hacen otros sitios exitosos. Si no nos mostramos seguros, pueden darle (parte de) el proyecto a otro equipo de desarrollo.
Probabilidad: Baja.
Impacto: Alto.
Exposición: Media.
Mitigación: Mostrar ejemplos de trabajos nuestros en los que no tuvimos esos problemas.
Plan de contingencia: ???

Riesgo 9
Descripción: Los distintos sponsors tienen intereses distintos, algunos de ellos contrapuestos. Por ejemplo, las empresas televisivas están en contra de las simulaciones. Puede ser que parte del proyecto cambie o se cancele.
Probabilidad: Baja.
Impacto: Medio.
Exposición: Baja.
Mitigación: Mantener reuniones periódicas con los sponsors y convencerlos de que ambas funcionalidades son buenas para el proyecto.
Plan de contingencia: Diseñar una arquitectura flexible, que se adapte a los cambios.

Riesgo 10
Descripción: Hay otros aspectos todavía no definidos y que no dependen de nosotros. Por ejemplo, en qué países se va a usar cada parte del juego. Puede pasar que cambien los requerimientos después de que empezamos.
Probabilidad: Media.
Impacto: Medio.
Exposición: Media.
Mitigación: Exigir a los stakeholders que tomen decisiones y las comuniquen lo antes posible.
Plan de contingencia: Diseñar una arquitectura flexible, que se adapte a los cambios.

Riesgo 11
Descripción: El equipo nunca trabajó con esta metodología. Puede pasar que la adaptación lleve más tiempo de lo planeado.
Probabilidad: Media.
Impacto: Medio.
Exposición: Alta.
Mitigación: Consultas periódicas con un experto.
Plan de contingencia: Aumentar las consultas al experto.

Riesgo 12
Descripción: El equipo tuvo problemas de comunicación antes. Si se repiten, pueden afectar todo el trabajo. Trabajamos mucho de manera remota y tenemos distintos horarios.
Probabilidad: Media.
Impacto: Alto.
Exposición: Alta.
Mitigación: Reuniones periódicas e intercambio de mails diarios.
Plan de contingencia: Reuniones de emergencia y buscar vías de comunicación instantáneas, como chats y teléfono.

Riesgo 13
Descripción: Es probable que la planificación no sea exacta, tanto en las tareas como en los tiempos pensados para cada una.
Probabilidad: Alta.
Impacto: Medio.
Exposición: Alta.
Mitigación: Sobreestimar los tiempos y las dificultades.
Plan de contingencia: Evaluar y adaptar la planificación luego de la primera iteración.


  \newpage
\section{Plan de Proyecto}
\label{sec:planificacion}
\subsection{Iteraciones}
Las iteraciones tienen una duración aproximada de 3 semanas (15 días hábiles). Teniendo en cuenta que los recursos asignados son 4 arquitectos de software/programadores trabajando de modo part-time (6 horas), se estiman unas 360hrs. aproximadas por iteración.

Nuestra decisión sobre la conformación de las iteraciones -en cuanto a los casos de uso- se rigió por varios factores: 
\begin{itemize}
\item Se tuvieron en cuenta las necesidades extraídas del QAW con los stakeholders
\item Se trató de agrupar a los casos de uso por su temática, prefiriendo en caso de ser posible juntar dentro de una misma iteración los casos de uso que se relacionan con funcionalidades similares.
\item Otro factor que se tuvo en cuenta a la hora de definir el orden de las iteraciones, fueron los riesgos detectados y analizados en la sección de Análisis de Riesgos (\ref{sec:riesgos}).
\item También se consideró la prioridad de la funcionalidades referidas en los casos de uso, intentando en lo posible desarrollar antes las más importantes para el negocio.
\end{itemize}

La primera iteración quedó expresamente definida por el QAW y el análisis de riesgo, factores a los que se les dio mayor prioridad. 
Del análisis de riesgo, y dado el factor prioritario que tiene la disponibilidad del sistema, se extrajo que la parte más conflictiva ocurre a la hora de transmitir los desafíos (especialmente para los eventos globales), así como al mostrar correctamente las simulaciones.

El siguiente punto en cuanto a nivel de riesgo, fue lo concerniente a la seguridad, integridad y transparencia de la transmisión de los datos de pago y movimientos de dinero, y las transmisiones de datos usadas para definir los resultados de los partidos de la realidad y minuto a minuto de las simulaciones. Decidimos encarar esto en la segunda iteración del proyecto.

Otro foco de preocupación y de vital importancia para el proyecto, es su alcance y monetización. Por esa razón, se decidió trabajar en lo que respecta a las publicidades del sitio, simulaciones y transmisiones (y el uso de los datos de comportamiento de usuarios), además de la regionalización del sistema (con todas las reglas y restricciones que eso conlleva) en la tercer iteración.

El manejo administrativo de las cuentas de los usuarios, y el aspecto \emph{social} (chat, utilización de las opiniones vertidas en redes sociales, etc.), consideradas cuestiones de menor complejidad, se decidieron enfrentar en la cuarta iteración.

Decidimos dedicarnos en la quinta iteración a todo lo referido a los desafíos, su creación, configuración, definición de reglas y de los premios. En la sexta iteración, nos encargaremos de lo que concierne a los rankings de los jugadores globales, regionales, etc. Como los casos de uso de estas dos iteraciones tienen un menor riesgo y complejidad que los anteriores, consideramos que estas dos fases son de construcción, ya que se trabajará bastante en lo implementativo y poco en el análisis/diseño.



\subsubsection{I01 - Primera Iteración (Elaboración)}
\begin{itemize}
\item (CU19) Mostrando detalle minuto a minuto de la simulación
\item (CU2)  Obteniendo datos en tiempo real
\item (CU21) Mostrando simulación gráfica
\item (CU22) Observando la transmisión de un partido de liga
\item (CU20) Observando evento global/continental
%\item CU15) Mostrando desarrollo (log o streaming) de desafío} (sistema -> usuario)
\end{itemize}

\subsubsection{I02 - Segunda iteración (Elaboración)} 
\begin{itemize}
\item (CU9) Creando cuenta de usuario
\item (CU10) Actualizando datos de medios de pago
\item (CU34) Auditando movimientos de dinero
\item (CU33) Consultando estado de cuenta y movimientos de usuario
\item (CU35) Auditando simulaciones
\item (CU8) Apostando
%\item CU43) Logueando movimientos de dinero}: (depende de lo que diga Javi)
\end{itemize}


\subsubsection{I03 - Tercera Iteración (Elaboración)}
\begin{itemize}
\item (CU36) Definiendo restricciones por zona
\item (CU25) Definiendo regiones de la plataforma
\item (CU18) Definiendo reglas de simulación
\item (CU38) Configurando publicidad y ads en transmisiones
\item (CU16) Configurando publicidad en el sitio y simulaciones
\item (CU17) Acceder a datos de preferencia/comportamiento de usuarios
%\item (CU44) Mostrando publicidad en la simulación} - ¿es algo que se hace al generar la simulación,  puede ser luego?
%\item (CU37) Configurando publicidades en simulaciones y en el sitio}:  (representante de empresas (?))
%\item (CU39) Consultando datos de usuario}: (representantes de empresas(?))
\end{itemize}


\subsubsection{I04 - Cuarta Iteración (Elaboración)}
\begin{itemize}
\item (CU6) Participando del chat general
\item (CU7) Participando de chat privado
\item (CU12) Consultando cuenta de usuario
\item (CU41) Desactivando cuenta
\item (CU42) Reactivando cuenta
\item (CU23) Recolectando opiniones de redes sociales y chats
\item (CU24) Definiendo impacto de opiniones
%4- consultando datos de jugadores /técnicos /equipos (participante)
\end{itemize}

\subsubsection{I05 - Quinta Iteración (Construcción)}
\begin{itemize}
\item (CU1) Eligiendo liga para competir
\item (CU3) Definiendo reglas de desafío
\item (CU4) Creando desafío
\item (CU5) Aceptando desafío
\item (CU14) Consultando estado (cuenta regresiva, participantes, posiciones) del desafío
\item (CU13) Definiendo premios
\end{itemize}

\subsubsection{I06 - Sexta Iteración (Construcción)}
\begin{itemize}
\item (CU28) Consultando dashboard regional o global
\item (CU29) Consultando ranking de jugadores
\item (CU32) Reiniciando el ranking de jugadores
\item (CU26) Definiendo desafíos interzonales
\item (CU31) Configurando visibilidad de los desafíos
%6 -  definiendo reglas de puntajes (administrador)
\end{itemize}


\subsection{Primera iteración (I01) en detalle}
\label{subsec:primeraiteracion}
En esta sección se realiza un detalle de las tareas que se consideran necesarias realizar en la primer iteración, junto a las horas hombres esperadas. Dado que ya se realizó un reconocimiento, priorización, y estimación de tiempo de los casos de uso, además del análisis de riesgo, no serán tareas prioritarias ni de mucha intensidad en esta etapa, sino que se utilizarán para poder corregir detalles de las futuras iteraciones.

Algunas de las tareas (por ejemplo I01T01, I01T02, I01T03) que no se corresponden directamente con un caso de uso, tienen como objetivo la mitigación de los riesgos analizados (\ref{sec:riesgos}).

\begin{itemize}
\item \underline{I01T01} - Reunión semanal breve con stakeholders para asegurarse que los requerimientos estén actualizados: \textbf{10hs}.
  \begin{itemize}
    \item \underline{I01T01ST01} - Reunión la primer semana: 4hs.
    \item \underline{I01T01ST02} - Reunión la segunda semana: 3hs.
    \item \underline{I01T01ST03} - Reunión la tercer semana: 3hs.
  \end{itemize}
\hfill

\item \underline{I01T02} - Reunión cada 3 días hábiles del equipo para evitar problemas de comunicación como en el pasado: \textbf{10hs}.
  \begin{itemize}
    \item \underline{I01T02ST01} - Primera reunión: 2hs.
    \item \underline{I01T02ST02} - Segunda reunión: 2hs.
    \item \underline{I01T02ST03} - Tercer reunión: 2hs.
    \item \underline{I01T02ST04} - Cuarta reunión: 2hs.
    \item \underline{I01T02ST05} - Quinta reunión: 2hs.
  \end{itemize}
\hfill

\item \underline{I01T03} - Reunión semanal de consulta con experto en metodología UP: \textbf{8hs}.
  \begin{itemize}
    \item \underline{I01T03ST01} - Reunión la primer semana: 3hs.
    \item \underline{I01T03ST02} - Reunión la segunda semana: 2hs.
    \item \underline{I01T03ST03} - Reunión la tercer semana: 2hs.
  \end{itemize}
\hfill

\item \underline{I01T04} - Identificación y descripción de los atributos de calidad del sistema: \textbf{40hs}.
  \begin{itemize}
    \item \underline{I01T04ST01} - Cotejamiento de los atributos definidos por stakeholders en QAW y relación con los casos de uso definidos: 16hrs.
    \item \underline{I01T04ST02} - Descripción de escenarios de atributos de calidad: 20hs.
    \item \underline{I01T04ST03} - Verificación de la documentación escrita: 4hs.
  \end{itemize}
\hfill
  
\item \underline{I01T05} - Diseño de la arquitectura del sistema: \textbf{60hs}.
  \begin{itemize}
    \item \underline{I01T05ST01} - Analizar los escenarios descritos en I01T04 e identificar drivers de arquitectura: 8hs. 
    \item \underline{I01T05ST02} - Estudiar y elegir patrones arquitectónicos que satisfagan los drivers identificados: 12hs.
    \item \underline{I01T05ST03} - Verificar y refinar los casos de usos y los escenarios: 10hs.
    \item \underline{I01T05ST04} - Iterar: 30hs.
   \end{itemize}
\hfill

\item \underline{I01T06} - Realización de las tareas del (CU2)  Obteniendo datos en tiempo real: \textbf{40hs}.
  \begin{itemize}
    \item \underline{I01T06ST01} - Reunirse con empresas que provean datos en tiempo real de la evolución de los partidos y contratar alguna: 12hs.
    \item \underline{I01T06ST02} - Reunirse con la empresa contratada y obtener documentación técnica sobre la API que proveen: 2hs.
    \item \underline{I01T06ST03} - Investigar la API y analizar qué datos podemos usar como input en nuestro sistema: 6hs.
    \item \underline{I01T06ST04} - Adaptar nuestro sistema para que los desafíos en modo \emph{Liga de fantasía} utilicen los datos obtenidos a través de la api: 10hs.
    \item \underline{I01T06ST05} - Realizar algunos desafíos y verificar que los resultados de los puntajes se condigan con lo que ocurrió en los partidos reales: 6hs.
    \item \underline{I01T06ST06} - Corregir potenciales errores: 4hs.
  \end{itemize}
\hfill

\item \underline{I01T07} - Realización de las tareas del (CU19) Mostrando detalle minuto a minuto de la simulación: \textbf{40hs}.
  \begin{itemize}
    \item \underline{I01T07ST01} - Investigar el log del desarrollo anterior, cómo se genera y la calidad de la salida: 4hs.
    \item \underline{I01T07ST02} - Reunirse con las empresas desarrolladoras de los dos motoroes gráficos y averiguar qué tipo de entrada necesitan: 6hs.
    \item \underline{I01T07ST03} - Comparar el log actual con el que se necesita y definir qué es lo que falta agregar: 4hs.
    \item \underline{I01T07ST04} - Agregar el detalle necesario al log de salida para cumplir con lo requerido por las empresas: 10hs.
    \item \underline{I01T07ST05} - Ajustar la velocidad de salida del log para que no sea instantánea, sino minuto a minuto, y respete la nueva duración (similar a la de un partido real) de las simulaciones: 5hs.
    \item \underline{I01T07ST06} - Realizar algunas ejecuciones de prueba y obtener logs de salida de ejemplo: 4hs.
    \item \underline{I01T07ST07} - Corroborar con empresas proveedoras que el detalle del log obtenido sea el correcto: 2hs.    
    \item \underline{I01T07ST08} - Corregir potenciales errores: 5hs.
  \end{itemize}
\hfill

\item \underline{I01T08} - Realización de las tareas del (CU21) Mostrando simulación gráfica: \textbf{50hs}.
  \begin{itemize}
    \item \underline{I01T08ST01} - Reunirse con empresa de motor 3D y discutir sobre los requerimientos técnicos para la transmisión de video en dispositivos no soportados por el motor gráfico: 10hs.
    \item \underline{I01T08ST02} - Obtener varios logs minuto a minuto de prueba y utilizarlos como entrada para los distintos motores gráficos. Comparar que el resultado gráfico (según limitaciones de cada motor) de los partidos sea el mismo y no haya diferencias: 20hs.
    \item \underline{I01T08ST03} - Probar los motores en distintos dispositivos y obtener los requerimientos mínimos y recomendados de hardware para poder informar a los participantes: 15hs.
    \item \underline{I01T08ST04} - Escribir documentación sobre los distintos motores, obtener screenshots para poder publicar en el home del producto como ejemplo de jugabilidad: 5hs.
  \end{itemize}
\hfill


\item \underline{I01T09} - Realización de las tareas del (CU22) Observando la transmisión de un partido de liga: \textbf{65hs}.
  \begin{itemize}
    \item \underline{I01T09ST01} - Reunirse con la empresa dueña de los derechos de televisión y la empresa proveedora de infraestructura para definir sus requerimientos, necesidades, y llegar a acuerdos comunes en los puntos álgidos: 10hs.
    \item \underline{I01T09ST02} - Reunirse con la empresa proveedora de infraestructura de redes para definir la arquitectura de hardware a utilizar para el sistema: 8hs.
    \item \underline{I01T09ST03} - Implementar la solución de hardware convenida con la empresa proveedora de infraestructura de redes: 16hs.
    \item \underline{I01T09ST04} - Realizar pruebas internas de transmisión en vivo de eventos con diferentes cargas en los servidores y desde distintas regiones: 16hs.
    \item \underline{I01T09ST05} - Dar feedback a la empresa proveedora de infraestructura y ver, de ser necesario, cómo mejorar el rendimiento: 10hs.
    \item \underline{I01T09ST06} - Mostrarle a la empresa dueña de los derechos de televisión el funcionamiento del sistema y verificar que cumpla con sus requerimientos: 5hs.
  \end{itemize}
\hfill

\item \underline{I01T10} - Realización de las tareas del (CU20) Observando evento global/continental: \textbf{hs}.
  \begin{itemize}
    \item \underline{I01T10ST01} - : XXhs.
    \item \underline{I01T10ST02} - : XXhs.
    \item \underline{I01T10ST03} - : XXhs.
    \item \underline{I01T10ST04} - : XXhs.
  \end{itemize}
\hfill

La suma de las tareas para la primer iteración da ???hs. Habiendo calculado una iteración ideal de 360hs. nos da cierto margen como para poder afrontar tareas urgentes o no previstas.

\subsection{Diagrama de Gantt}
  
\end{itemize}

  \newpage
\section{Atributos de calidad}

En esta sección se presentan los atributos de calidad identificados en el TP (ordenados según la prioridad definida por los \emph{stakeholders} en el \emph{Quality Attribute Workshop}), junto a sus respectivos escenarios.

% \begin{center}
%   \begin{tabular}{| l | p{10cm} | }
%     \hline
% 	\textbf{Descripción} & UNA DESCRIPCION\\  \hline
% 	\textbf{Fuente} & UNA FUENTE\\  \hline
% 	\textbf{Estímulo} & UN ESTIMULO\\  \hline
% 	\textbf{Entorno} & UN ENTORNO\\  \hline
% 	\textbf{Artefacto} & UN ARTEFACTO\\  \hline
% 	\textbf{Respuesta} & UNA RESPUESTA\\  \hline
% 	\textbf{Medición} & UNA MEDICION\\  \hline
%   \end{tabular}
% \end{center} 

\subsection{Disponibilidad}

\subsubsection*{Motivación}
\begin{itemize}
\item El sistema tiene que estar andando todo el tiempo.
\item Cantidad de usuarios limitada - máxima por servidor
\item Múltiples nodos distribuidos de pocos usuarios cada uno.
\item Los partidos streameados deben darse en excelente calidad sin cortes.
\item Cuando ninguno de los motores gráficos se encuentra disponible, debe usarse el streaming del engine 3D.
\item Debe esforzarse por respetar a los países en los que la legislación vigente no permite directamente que se ingrese al sitio.
\item Debe ser fácil desactivar una cuenta por un tiempo, sin poder reactivarla, tanto para usuarios adictos en recuperación como para usuarios de países en los que no está disponible el sitio.
\end{itemize}

\subsubsection*{Escenarios}
\begin{center}
  \begin{tabular}{| l | p{10cm} | }
    \hline
	\textbf{Descripción} & En caso de saturación de un nodo, deben direccionarse nuevas conexiones a otro nodo.\\  \hline
	\textbf{Fuente} & Usuario\\ \hline
	\textbf{Estímulo} & Se realiza un request al sistema.\\  \hline
	\textbf{Entorno} & Nodo saturado\\  \hline
	\textbf{Artefacto} & Nodo del sistema\\  \hline
	\textbf{Respuesta} & Se selecciona un nuevo nodo regional y se continúa el servicio mediante una conexión con él.\\  \hline
	\textbf{Medición} & El 99.99 \% de las nuevas conexiones se realizan sin inconveniente alguno y de forma transparente al usuario.\\  \hline
  \end{tabular}
\end{center} 


\begin{center}
  \begin{tabular}{| l | p{10cm} | }
    \hline
	\textbf{Descripción} & Los streamings de partidos se realizan con la menor cantidad de cortes posibles.\\  \hline
	\textbf{Fuente} & Usuario\\  \hline
	\textbf{Estímulo} & Solicita ver el streaming de un partido\\  \hline
	\textbf{Entorno} & Normal\\  \hline
	\textbf{Artefacto} & Generador gráfico\\  \hline
	\textbf{Respuesta} & Se realizan los cálculos pertinentes para determinar el ancho de banda del usuario y se comienza el streaming con la calidad determinada.\\  \hline
	\textbf{Medición} & Se produce a lo sumo un corte con duración menor a 5 segundos por transmisión.\\  \hline
  \end{tabular}
\end{center}

\begin{center}
  \begin{tabular}{| l | p{10cm} | }
    \hline
	\textbf{Descripción} & Si los motores gráficos para una simulación no están disponibles, debe realizarse un streaming del engine 3D para la simulación. \\  \hline
	\textbf{Fuente} & Usuario \\  \hline
	\textbf{Estímulo} & Intenta observar una simulación\\  \hline
	\textbf{Entorno} & Motores gráficos del dispositivo no disponibles.\\  \hline
	\textbf{Artefacto} & Generador gráfico\\  \hline
	\textbf{Respuesta} & Ante la detección de la imposibilidad de utilizar los motores gráficos en el dispositivo, se activa el streaming del engine 3D de la simulación.\\  \hline
	\textbf{Medición} & Se trasmite la simulación 3D sin inconvenientes en el 99\% de los casos.\\  \hline
  \end{tabular}
\end{center}  

\begin{center}
  \begin{tabular}{| l | p{10cm} | }
    \hline
	\textbf{Descripción} & Si se detecta que los motores gráficos vuelven a estar disponibles, se los vuelve a utilizar en lugar del streaming de la simulación \\  \hline
	\textbf{Fuente} & Identificador de dispositivo\\  \hline
	\textbf{Estímulo} & Detecta motores gráficos disponibles\\  \hline
	\textbf{Entorno} & Streaming de una simulación\\  \hline
	\textbf{Artefacto} & Generador gráfico\\  \hline
	\textbf{Respuesta} & Se utiliza alguno de los motores gráficos del dispositivo\\  \hline
	\textbf{Medición} & Se corta el streaming de video y en menos de 10 segundos se continúa la simulación con el motor gráfico.\\  \hline
  \end{tabular}
\end{center}  

\begin{center}
  \begin{tabular}{| l | p{10cm} | }
    \hline
	\textbf{Descripción} & Si un usuario idenficado como adicto en recuperación intenta conectarse, se le niega el acceso. \\  \hline
	\textbf{Fuente} & Usuario identificado como adicto\\  \hline
	\textbf{Estímulo} & Intento de conexión\\  \hline
	\textbf{Entorno} & Normal\\  \hline
	\textbf{Artefacto} & Controlador de Alta de Usuarios\\  \hline
	\textbf{Respuesta} & Se le niega el acceso al sistema.\\  \hline
	\textbf{Medición} & En el 99.9999 \% de los casos de intento de conexión proveniente de un usuario identificado como adicto en recuperación, se le niega el acceso al sistema con un cartel informativo.\\  \hline
  \end{tabular}
\end{center}  

\begin{center}
  \begin{tabular}{| l | p{10cm} | }
    \hline
	\textbf{Descripción} & Se bloquea el acceso desde países que prohiben el uso del sistema.\\  \hline
	\textbf{Fuente} & Usuario de país que prohibe el uso del sistema\\  \hline
	\textbf{Estímulo} & Intento de conexión\\  \hline
	\textbf{Entorno} & Normal\\  \hline
	\textbf{Artefacto} & Controlador de Alta de Usuarios\\  \hline
	\textbf{Respuesta} & Se le niega el acceso al sistema.\\  \hline
	\textbf{Medición} & En el 99.9999 \% de los casos de intento de conexión proveniente de un país que no permite el acceso al sitio, se le niega el accesso y se le muestra un cartel informativo.\\  \hline
  \end{tabular}
\end{center} 


\subsection{Seguridad}

\subsubsection*{Motivación}
\begin{itemize}
\item Todo lo relativo al manejo del dinero (depósito y retiro de los participantes vía tarjeta de crédito o caja de ahorro) deber ser seguro y transparente.
\item Se teme por la seguridad de los datos de los usuarios, tanto por el robo de la información de las tarjetas de crédito o cajas de ahorro como de la explotación de los datos de los usuarios por terceros para fines no autorizados.
\item Preocupa el resultado de los desafíos, por el pago/cobro a los participantes, y la coherencia con los datos provenientes de las empresas que relevan los resultados de los partidos.
\item Es importante proveer transparencia acerca del funcionamiento de las simulaciones. 
\item Los módulos de las simulaciones deben poder ser inspeccionados fácilmente por entidades de control.
\item Se debe loggear todo movimiento de dinero para evitar cualquier tipo de evasión impositiva.
\end{itemize}

\subsubsection*{Escenarios}
\begin{center}
  \begin{tabular}{| l | p{10cm} | }
    \hline
	\textbf{Descripción} & La información de los medios de pago de los usuarios está protegida contra el robo de datos\\  \hline
	\textbf{Fuente} & Atacante externo\\  \hline
	\textbf{Estímulo} & Intenta vulnerar la información de medios de pago de los usuarios mediante el descifrado de los datos.\\  \hline
	\textbf{Entorno} & Normal\\  \hline
	\textbf{Artefacto} & Datos del Almacén de información de crédito \\  \hline
	\textbf{Respuesta} & Los datos en texto plano no son accesibles al atacante en tiempos razonables.\\  \hline
	\textbf{Medición} & En el 99.9999 \% de los casos, los ataques no son exitosos.\\  \hline
  \end{tabular}
\end{center}

\begin{center}
  \begin{tabular}{| l | p{10cm} | }
    \hline
	\textbf{Descripción} & Los movimientos de dinero se registran en el sistema\\  \hline
	\textbf{Fuente} & Usuario o Gestor de crédito\\  \hline
	\textbf{Estímulo} & Se apuesta dinero o se recibe un pago.\\  \hline
	\textbf{Entorno} & Normal\\  \hline
	\textbf{Artefacto} & Gestor de crédito\\  \hline
	\textbf{Respuesta} & Se realiza correctamente el movimiento y se loggea un registro con los datos de la transacción\\  \hline
	\textbf{Medición} & El registro se guarda correctamente en el 99.999 \% de los casos.\\  \hline
  \end{tabular}
\end{center}  

\begin{center}
  \begin{tabular}{| l | p{10cm} | }
    \hline
	\textbf{Descripción} & Un usuario no puede consultar los datos de pago que no le corresponden.\\  \hline
	\textbf{Fuente} & Usuario\\  \hline
	\textbf{Estímulo} & Intenta obtener los datos de pago de otro usuario.\\  \hline
	\textbf{Entorno} & Normal\\  \hline
	\textbf{Artefacto} & Gestor de crédito\\  \hline
	\textbf{Respuesta} & Se rechaza el intento y se loggea un registro de la solicitud.\\  \hline
	\textbf{Medición} & La solicitud se rechaza un 99.99 \% de las veces.\\  \hline
  \end{tabular}
\end{center}

\begin{center}
  \begin{tabular}{| l | p{10cm} | }
    \hline
	\textbf{Descripción} & Los resultados de los desafíos en modo Liga de Fantasía, se corresponden con los datos provistos por las empresas relevadoras de datos de los partidos\\  \hline
	\textbf{Fuente} & Empresa relevadora de datos de partidos\\  \hline
	\textbf{Estímulo} & Se envían resultados minuto a minuto de un partido\\  \hline
	\textbf{Entorno} & Normal\\  \hline
	\textbf{Artefacto} & Controlador de desafíos\\  \hline
	\textbf{Respuesta} & Se autentica la identidad de la empresa relevadora de datos y se persisten los resultados en el sistema.\\  \hline
	\textbf{Medición} & En más del 99\% de los casos los resultados de los desafíos se contrastan correctamente con los datos provistos por las empresas.\\  \hline
  \end{tabular}
\end{center}

\begin{center}
  \begin{tabular}{| l | p{10cm} | }
    \hline
	\textbf{Descripción} & Se rechazan los datos de los partidos de fuentes no autenticadas correctamente\\  \hline
	\textbf{Fuente} & Atacante externo\\  \hline
	\textbf{Estímulo} & Se hace pasar por una de las empresas contratadas para proveer los datos y de esta forma alterar los resultados de los desafíos\\  \hline
	\textbf{Entorno} & Normal\\  \hline
	\textbf{Artefacto} & Controlador de desafíos\\  \hline
	\textbf{Respuesta} & Se detecta la intrusión al no poder autenticar la identidad de la fuente de datos, se rechazan los datos y se registra el potencial ataque para futuros análisis\\  \hline
	\textbf{Medición} & En más del 99.99\% de los casos el ataque es correctamente detectado y repelido.\\  \hline
  \end{tabular}
\end{center}    


\begin{center}
  \begin{tabular}{| l | p{10cm} | }
    \hline
	\textbf{Descripción} & Un usuario puede acceder a sus datos de medios de pago\\  \hline
	\textbf{Fuente} & Usuario\\  \hline
	\textbf{Estímulo} & Intenta consultar o modificar sus datos de medios de pago\\  \hline
	\textbf{Entorno} & Normal\\  \hline
	\textbf{Artefacto} & Controlador de Alta de Medios de pago\\  \hline
	\textbf{Respuesta} & Se autoriza el acceso.\\  \hline
	\textbf{Medición} & La solicitud se autoriza 99.9999 \% de las veces.\\  \hline
  \end{tabular}
\end{center} 


%%% REPETIDO!
%\begin{center}
%  \begin{tabular}{| l | p{10cm} | }
%    \hline
%	\textbf{Descripción} & Un usuario puede acceder a sus datos de medios de pago\\  \hline
%	\textbf{Fuente} & Usuario\\  \hline
%	\textbf{Estímulo} & Intenta consultar o modificar sus datos de medios de pago\\  \hline
%	\textbf{Entorno} & Normal\\  \hline
%	\textbf{Artefacto} & Controlador de Medios de pago\\  \hline
%	\textbf{Respuesta} & Se autoriza el acceso.\\  \hline
%	\textbf{Medición} & La solicitud se autoriza 99.99 \% de las veces.\\  \hline
%  \end{tabular}
%\end{center}


\begin{center}
  \begin{tabular}{| l | p{10cm} | }
    \hline
	\textbf{Descripción} & El funcionamiento de los módulos de simulación es el esperado\\  \hline
	\textbf{Fuente} & Auditor\\  \hline
	\textbf{Estímulo} & Prueba los módulos en servidores testigo\\  \hline
	\textbf{Entorno} & Entorno de prueba\\  \hline
	\textbf{Artefacto} & Calculador de jugadas\\  \hline
	\textbf{Respuesta} & Se ejecutan los casos de prueba de los sistemas testigos.\\  \hline
	\textbf{Medición} & Se superan todos los casos de prueba de los sistemas testigos.\\  \hline
  \end{tabular}
\end{center}

\begin{center}
  \begin{tabular}{| l | p{10cm} | }
    \hline
	\textbf{Descripción} & El software instalado en los servidores del sistema se corresponde con el realizado por el equipo de desarrollo.\\  \hline
	\textbf{Fuente} & Auditor y equipo de desarrollo\\  \hline
	\textbf{Estímulo} & Se ejecutan mecanismos de hashing para comprar los componentes.\\  \hline
	\textbf{Entorno} & Entorno de prueba\\  \hline
	\textbf{Artefacto} & Sistema\\  \hline
	\textbf{Respuesta} & Se obtiene el hash del software instalado en los servidores.\\  \hline
	\textbf{Medición} & En todos los casos el hash obtenido es igual al hash esperado por el equipo de desarrollo.\\  \hline
  \end{tabular}
\end{center}  




\subsection{Modificabilidad}

\subsubsection*{Motivación}
\begin{itemize}
\item Se deben poder agregar fácilmente nuevos deportes
\item Debe resultar fácil modificar las simulaciones con el fin de mejorarlas.
\item Los datos del sitio deben ser fácilmente minados, y además debe poder crearse fácilmente reportes a partir de ellos.
\item Se deben poder modificar fácilmente las publicidades, tanto de las simulaciones como del sistema.
\end{itemize}

\subsubsection*{Escenarios}
\begin{center}
  \begin{tabular}{| l | p{10cm} | }
    \hline
	\textbf{Descripción} & Es fácil agregar un nuevo deporte al sistema.\\  \hline
	\textbf{Fuente} & Desarrollador\\  \hline
	\textbf{Estímulo} & Introduce un nuevo deporte al sistema.\\  \hline
	\textbf{Entorno} & Tiempo de desarrollo y diseño\\  \hline
	\textbf{Artefacto} & Sistema\\  \hline
	\textbf{Respuesta} & Los cambios se agregan sin ninguna complicación y se pueden realizar simulaciones del nuevo deporte.\\  \hline
	\textbf{Medición} & En menos de 40 horas hombre debe poder realizarse la integración del nuevo deporte al sistema sin afectar negativamente a las funcionalidades previamente provistas. El grueso del costo temporal se dedica al modelado y relevado del deporte en cuestión y no a la integración en el sistema.\\  \hline
  \end{tabular}
\end{center}

\begin{center}
  \begin{tabular}{| l | p{10cm} | }
    \hline
	\textbf{Descripción} & Se pueden modificar fácilmente las simulaciones para mejorarlas.\\  \hline
	\textbf{Fuente} & Desarrollador\\  \hline
	\textbf{Estímulo} & Introduce cambios en el módulo de simulación.\\  \hline
	\textbf{Entorno} & Tiempo de desarrollo y diseño\\  \hline
	\textbf{Artefacto} & Sistema\\  \hline
	\textbf{Respuesta} & Los cambios se agregan sin ninguna complicación, y la nueva versión del simulador inmediatamente se pone en marcha.\\  \hline
	\textbf{Medición} & En menos de 50 horas hombre se pueden agregar cambios a los módulos de simulación.\\  \hline
  \end{tabular}
\end{center}

\begin{center}
  \begin{tabular}{| l | p{10cm} | }
    \hline
	\textbf{Descripción} & Se pueden agregar nuevos reportes de manera fácil y rápida.\\  \hline
	\textbf{Fuente} & Desarrollador\\  \hline
	\textbf{Estímulo} & Intorudce un nuevo reporte.\\  \hline
	\textbf{Entorno} & Tiempo de desarrollo y diseño\\  \hline
	\textbf{Artefacto} & Sistema\\  \hline
	\textbf{Respuesta} & Se realiza el reporte y se lo agrega al sistema sin ningún incoveniente.\\  \hline
	\textbf{Medición} & En menos de 30 horas hombres se puede generar un nuevo reporte con los datos persistidos en el sistema.\\  \hline
  \end{tabular}
\end{center}

\subsection{Performance}

\subsubsection*{Motivación}
\begin{itemize}
\item Se requiere que las transmisiones de los partidos se realicen con excelente calidad y sin cortes.
\item Las simulaciones deben realizarse en tiempo real minuto a minuto.
\item Los eventos globales deben funcionar correctamente.
\end{itemize}

\subsubsection*{Escenarios}
\begin{center}
  \begin{tabular}{| l | p{10cm} | }
    \hline
	\textbf{Descripción} & En lo posible las simulaciones se grafican en los dispositivos del usuario y no en el sistema.\\ \hline
	\textbf{Fuente} & Usuario\\  \hline
	\textbf{Estímulo} & Visualiza una simulación.\\  \hline
	\textbf{Entorno} & Normal\\  \hline
	\textbf{Artefacto} & Graficador de simulaciones\\  \hline
	\textbf{Respuesta} & Se envían solamente los datos necesarios para que el engine opere en el dispositivo del cliente.\\ \hline
	\textbf{Medición} & En 9 de cada 10 dispositivos se grafica la simulación en alguno de los dos motores gráficos en vez de streamear el engine 3D.\\  \hline
  \end{tabular}
\end{center} 

\begin{center}
  \begin{tabular}{| l | p{10cm} | }
    \hline
	\textbf{Descripción} & Se puede consumir la transmisión de un partido de manera rápida y con buena calidad.\\  \hline
	\textbf{Fuente} & Usuario\\  \hline
	\textbf{Estímulo} & Visualiza un partido y se detecta transmisión no óptima.\\  \hline
	\textbf{Entorno} & Ancho de banda del usuario limitado\\  \hline
	\textbf{Artefacto} & Streameador de videos\\  \hline
	\textbf{Respuesta} & Se ejecuta los calculos pertinentes para acomodar el bitrate del video de manera tal que se ajuste acordemente al ancho de banda disponible para el usuario\\  \hline
	\textbf{Medición} & En el 99 \% de los casos, el usuario puede visualizar un partido de manera fluida esperando. El tiempo de espera para el comienzo del streaming de video es menor a 20 segundos.\\  \hline
  \end{tabular}
\end{center}

\begin{center}
  \begin{tabular}{| l | p{10cm} | }
    \hline
	\textbf{Descripción} & Los eventos globales se visualizan correctamente\\  \hline
	\textbf{Fuente} & Usuario\\  \hline
	\textbf{Estímulo} & Visualización de un evento global\\  \hline
	\textbf{Entorno} & Sistema con altos índices de demanda para la visualización del evento.\\  \hline
	\textbf{Artefacto} & Streameador de videos\\  \hline
	\textbf{Respuesta} & De acuerdo a los anchos de banda de los usuarios se transmite el evento de la mejor manera posible.\\  \hline
	\textbf{Medición} & Más del 99 \% de los usuarios pueden visualizar el evento correctamente sin cortes y con tiempos de espera menores al minuto para comenzar la visualización.\\  \hline
  \end{tabular}
\end{center}    


\subsection{Usabilidad}

\subsubsection*{Motivación}
\begin{itemize}
\item{Debe resultar simple y rápido crear equipos y participar de desafíos}
\item{Los administradores deben poder acceder a un dashboard en tiempo real de estado de cuenta regional del sitio y de los grupos de usuarios y analizarlo fácilmente}
\end{itemize}

\subsubsection*{Escenarios}
\begin{center}
  \begin{tabular}{| l | p{10cm} | }
    \hline
	\textbf{Descripción} & Es fácil y rápido crear un equipo.\\  \hline
	\textbf{Fuente} & Usuario\\  \hline
	\textbf{Estímulo} & Intenta crear un nuevo equipo.\\  \hline
	\textbf{Entorno} & Normal\\  \hline
	\textbf{Artefacto} & Interfaz de Usuario\\  \hline
	\textbf{Respuesta} & Se crea correctamente un equipo.\\  \hline
	\textbf{Medición} & En un tiempo promedio de 20 minutos el usuario puede crear un equipo a su parecer.\\  \hline
  \end{tabular}
\end{center}


\begin{center}
  \begin{tabular}{| l | p{10cm} | }
    \hline
	\textbf{Descripción} & Un administrador puede acceder al dashboard de estado de cuenta y de los grupos de participantes de manera rápida y simple.\\  \hline
	\textbf{Fuente} & Administrador\\  \hline
	\textbf{Estímulo} & Accede al dashboard\\  \hline
	\textbf{Entorno} & Normal\\  \hline
	\textbf{Artefacto} & Interfaz de Administrador\\  \hline
	\textbf{Respuesta} & Se muestran el dashboard y sus opciones\\  \hline
	\textbf{Medición} & En menos de de un minuto el administrador puede consultar la información que requiere del dashboard.\\  \hline
  \end{tabular}
\end{center}
  \newpage 

\end{document}
